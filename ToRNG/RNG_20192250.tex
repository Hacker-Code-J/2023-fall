\documentclass[12pt,openany]{book}

\usepackage{amsmath,amsthm,amsfonts,amscd} % Packages for mathematics

% Colors
\usepackage[dvipsnames]{xcolor}
\definecolor{titleblue}{RGB}{0,53,128}
\definecolor{chaptergray}{RGB}{140,140,140}
\definecolor{sectiongray}{RGB}{180,180,180}

\definecolor{thmcolor}{RGB}{231, 76, 60}
\definecolor{defcolor}{RGB}{52, 152, 219}
\definecolor{lemcolor}{RGB}{155, 89, 182}
\definecolor{corcolor}{RGB}{46, 204, 113}
\definecolor{procolor}{RGB}{241, 196, 15}

% Fonts
\usepackage[T1]{fontenc}
\usepackage[utf8]{inputenc}
\usepackage{newpxtext,newpxmath}
\usepackage{sectsty}
\allsectionsfont{\sffamily\color{titleblue}\mdseries}

% Page layout
\usepackage{geometry}
\geometry{a4paper,left=1.325in,right=1in,top=1in,bottom=1in,heightrounded}
\usepackage{fancyhdr}
\fancyhf{}
\fancyhead[LE,RO]{\thepage}
\fancyhead[LO]{\nouppercase{\rightmark}}
\fancyhead[RE]{\nouppercase{\leftmark}}
\renewcommand{\headrulewidth}{0.5pt}
\renewcommand{\footrulewidth}{0pt}

% Chapter formatting
\usepackage{titlesec}
\titleformat{\chapter}[display]
{\normalfont\sffamily\Huge\bfseries\color{titleblue}}{\chaptertitlename\ \thechapter}{20pt}{\Huge}
\titleformat{\section}
{\normalfont\sffamily\Large\bfseries\color{titleblue!100!gray}}{\thesection}{1em}{}
\titleformat{\subsection}
{\normalfont\sffamily\large\bfseries\color{titleblue!75!gray}}{\thesubsection}{1em}{}

% Table of contents formatting
\usepackage{tocloft}
\renewcommand{\cftchapfont}{\sffamily\color{titleblue}\bfseries}
\renewcommand{\cftsecfont}{\sffamily\color{chaptergray}}
\renewcommand{\cftsubsecfont}{\sffamily\color{sectiongray}}
\renewcommand{\cftchapleader}{\cftdotfill{\cftdotsep}}

% Hyperlinks
\usepackage{hyperref}
\hypersetup{
	colorlinks=true,
	linkcolor=titleblue,
	filecolor=black,      
	urlcolor=titleblue,
}

%Listing
\usepackage{listings} %Code
\renewcommand{\lstlistingname}{Code}%

\definecolor{sagegreen}{rgb}{0.0,0.6,0.4}
\definecolor{sagepurple}{rgb}{0.6,0.0,0.4}
\definecolor{sageblue}{rgb}{0.0,0.4,0.6}
\definecolor{sageorange}{rgb}{1.0,0.4,0.0}
\definecolor{sagegray}{rgb}{0.4,0.4,0.4}

\lstdefinestyle{sage}{
	language=Python,
	backgroundcolor=\color{white},
	basicstyle=\small\ttfamily\color{black}, 
	basicstyle=\footnotesize\ttfamily\color{black},
	keywordstyle=\color{blue!60!black},
	commentstyle=\color{green!60!black},
	stringstyle=\color{purple!60!black},
	showstringspaces=false,
	breaklines=true,
	tabsize=4,
	morekeywords={True, False, None},
	frame=leftline, % Remove the border
	framesep=3pt,
	frameround=tttt,
	framexleftmargin=3pt,
	numbers=left,
	numberstyle=\small\color{gray},
	xleftmargin=15pt, % Increase the left margin
	xrightmargin=5pt,
	captionpos=b,
	belowskip=0pt,
	aboveskip=4pt
}

%Ceiling and Floor Function
\usepackage{mathtools}
\DeclarePairedDelimiter{\ceil}{\lceil}{\rceil}
\DeclarePairedDelimiter{\floor}{\lfloor}{\rfloor}

%Algorithm
\usepackage[ruled,linesnumbered]{algorithm2e}
\usepackage{setspace}
\usepackage{algpseudocode}
\SetKwComment{Comment}{/* }{ */}
\SetKw{Break}{break}
\SetKw{Downto}{downto}
\SetKwProg{Fn}{Function}{:}{end}
\SetKwFunction{KeyGen}{KeyGen}


%---------------------------My Preamble
\usepackage{marvosym} %Lightning
\usepackage{booktabs}
\usepackage{multicol}
\setlength{\columnsep}{2cm}
\setlength{\columnseprule}{1.25pt}
\usepackage{enumerate}
\usepackage{soul}
\newcommand{\mathcolorbox}[2]{\colorbox{#1}{$\displaystyle #2$}}
\usepackage{graphicx}
\usepackage{tikz}
\usepackage{tikz-cd}
\usetikzlibrary{calc}
\usetikzlibrary{arrows, arrows.meta, positioning, shapes.multipart}

%Tcolorbox
\usepackage[most]{tcolorbox}
\tcbset{colback=white, arc=5pt}
%\tcbset{enhanced, colback=white,colframe=black,fonttitle=\bfseries,arc=4mm,boxrule=1pt,shadow={2mm}{-1mm}{0mm}{black!50}}
%White box with black text and shadow
%\begin{tcolorbox}[colback=white,colframe=black,fonttitle=\bfseries,title=Black Shadow Box,arc=4mm,boxrule=1pt,shadow={2mm}{-1mm}{0mm}{black!50}]
%	This is a white box with black text and a subtle shadow. The shadow adds some depth and dimension to the box without overpowering the design.
%\end{tcolorbox}

%Theorem
\newtheorem{axiom}{Axiom}[chapter]
\newtheorem{theorem}{Theorem}[chapter]
\newtheorem{proposition}[theorem]{Proposition}
\newtheorem{corollary}{Corollary}[theorem]
\newtheorem{lemma}[theorem]{Lemma}

\theoremstyle{definition}
\newtheorem{definition}{Definition}[chapter]
\newtheorem{remark}{Remark}[chapter]
\newtheorem{exercise}{Exercise}[chapter]
\newtheorem{example}{Example}[chapter]
\newtheorem*{note}{Note}

%New Command
\newcommand{\set}[1]{\left\{#1\right\}}
\newcommand{\N}{\mathbb{N}}
\newcommand{\Z}{\mathbb{Z}}
\newcommand{\Q}{\mathbb{Q}}
\newcommand{\R}{\mathbb{R}}
\newcommand{\C}{\mathbb{C}}
\newcommand{\F}{\mathbb{F}}

% Begin document
\begin{document}
	
	% Title page
	\begin{titlepage}
		\begin{center}
			{\Huge\textsf{\textbf{Theory of Random Number Generation}}\par}
			\vspace{0.5in}
			{\Large Ji Yong-Hyeon\par}
			\vspace{1in}
			\includegraphics[width=2.5in]{rng.jpg}\par
			\vspace{1in}
			{\bf Department of Information Security, Cryptology, and Mathematics\par}
			{College of Science and Technology\par}
			{Kookmin University\par}
			%\includegraphics[width=1.5in]{school_logo.jpg}\par
			\vspace{.25in}
			{\large \today\par}
		\end{center}
	\end{titlepage}
	
	% Table of contents
	\tableofcontents
	
	% Chapters
	\mainmatter
	
	\chapter{Introduction}
	\begin{tcolorbox}[colback=white,colframe=defcolor,arc=5pt,title={\color{white}\bf Summary}]
		 \begin{itemize}
		 	\item Required Properties for Random Bit Generator
		 	\begin{itemize}
		 		\item \textbf{Unpredictability}, \textbf{Unbiasedness}, \textbf{Independence}
		 	\end{itemize}
		 	\item Components of Cryptographically Secure Random Bit Generator
		 	\begin{itemize}
		 		\item TRNG (Entropy Source) + PRNG (Pseudorandom Number Generator)
		 	\end{itemize}
		 	\item Methods for Evaluating the Security of Random Bit Generator
		 	\begin{itemize}
		 		\item Estimation of entropy for the output sequence from TRNG
		 		\item Statistical randomness tests for the output sequence from RNG
		 	\end{itemize}
		 	\item Types of Random Bit Generators
		 	\begin{itemize}
		 		\item Hardware/Software-based Random Bit Generators
		 		\item Operating System-based Random Bit Generators
		 		\item Various Standard Pseudorandom Number Generators
		 	\end{itemize}
		 	\end{itemize}
	\end{tcolorbox}
	 Functions of RBG (Random Bit Generator)
	
	Provides random numbers required for cryptographic systems
	An essential element (algorithm) for the operation of cryptographic systems and modules
	 Required Properties: Unpredictability, Unbiasedness, Independence between bits
	
	Ideally, the output should be akin to the results of "coin tossing."
	 Applications of Random Bit Generator
	
	Generation of Key and Initialization Vector (IV) used in symmetric-key cryptography (block/stream ciphers)
	Generation of various parameters in public-key cryptography: prime number generation, probabilistic public-key cryptography, etc.
	Generation of various parameters used in cryptographic protocols: nonce, salt, etc.
	
	
	% End document
\end{document}
