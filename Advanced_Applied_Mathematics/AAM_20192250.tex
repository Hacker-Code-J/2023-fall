\documentclass[12pt,openany]{book}

\usepackage{amsmath,amsthm,amsfonts,amscd} % Packages for mathematics
\usepackage{commath}

% Colors
\usepackage[dvipsnames]{xcolor}
\definecolor{titleblue}{RGB}{0,53,128}
\definecolor{chaptergray}{RGB}{140,140,140}
\definecolor{sectiongray}{RGB}{180,180,180}

\definecolor{thmcolor}{RGB}{231, 76, 60}
\definecolor{defcolor}{RGB}{52, 152, 219}
\definecolor{lemcolor}{RGB}{155, 89, 182}
\definecolor{corcolor}{RGB}{46, 204, 113}
\definecolor{procolor}{RGB}{241, 196, 15}

% Fonts
\usepackage[T1]{fontenc}
\usepackage[utf8]{inputenc}
\usepackage{newpxtext,newpxmath}
\usepackage{sectsty}
\allsectionsfont{\sffamily\color{titleblue}\mdseries}

% Page layout
\usepackage{geometry}
\geometry{a4paper,left=1.2in,right=.7in,top=1in,bottom=1in,heightrounded}
\usepackage{fancyhdr}
\fancyhf{}
\fancyhead[LE,RO]{\thepage}
\fancyhead[LO]{\nouppercase{\rightmark}}
\fancyhead[RE]{\nouppercase{\leftmark}}
\renewcommand{\headrulewidth}{0.5pt}
\renewcommand{\footrulewidth}{0pt}

% Chapter formatting
\usepackage{titlesec}
\titleformat{\chapter}[display]
{\normalfont\sffamily\Huge\bfseries\color{titleblue}}{\chaptertitlename\ \thechapter}{20pt}{\Huge}
\titleformat{\section}
{\normalfont\sffamily\Large\bfseries\color{titleblue!100!gray}}{\thesection}{1em}{}
\titleformat{\subsection}
{\normalfont\sffamily\large\bfseries\color{titleblue!75!gray}}{\thesubsection}{1em}{}

% Table of contents formatting
\usepackage{tocloft}
\renewcommand{\cftchapfont}{\sffamily\color{titleblue}\bfseries}
\renewcommand{\cftsecfont}{\sffamily\color{chaptergray}}
\renewcommand{\cftsubsecfont}{\sffamily\color{sectiongray}}
\renewcommand{\cftchapleader}{\cftdotfill{\cftdotsep}}

% Hyperlinks
\usepackage{hyperref}
\hypersetup{
	colorlinks=true,
	linkcolor=titleblue,
	filecolor=black,      
	urlcolor=titleblue,
}

%Listing
\usepackage{listings} %Code
\renewcommand{\lstlistingname}{Code}%

\definecolor{sagegreen}{rgb}{0.0,0.6,0.4}
\definecolor{sagepurple}{rgb}{0.6,0.0,0.4}
\definecolor{sageblue}{rgb}{0.0,0.4,0.6}
\definecolor{sageorange}{rgb}{1.0,0.4,0.0}
\definecolor{sagegray}{rgb}{0.4,0.4,0.4}

\lstdefinestyle{sage}{
	language=Python,
	backgroundcolor=\color{white},
	basicstyle=\small\ttfamily\color{black}, 
	basicstyle=\footnotesize\ttfamily\color{black},
	keywordstyle=\color{blue!60!black},
	commentstyle=\color{green!60!black},
	stringstyle=\color{purple!60!black},
	showstringspaces=false,
	breaklines=true,
	tabsize=4,
	morekeywords={True, False, None},
	frame=leftline, % Remove the border
	framesep=3pt,
	frameround=tttt,
	framexleftmargin=3pt,
	numbers=left,
	numberstyle=\small\color{gray},
	xleftmargin=15pt, % Increase the left margin
	xrightmargin=5pt,
	captionpos=b,
	belowskip=0pt,
	aboveskip=4pt
}

%Ceiling and Floor Function
\usepackage{mathtools}
\DeclarePairedDelimiter{\ceil}{\lceil}{\rceil}
\DeclarePairedDelimiter{\floor}{\lfloor}{\rfloor}

%Algorithm
\usepackage[ruled,linesnumbered]{algorithm2e}
\usepackage{setspace}
\usepackage{algpseudocode}
\SetKwComment{Comment}{/* }{ */}
\SetKw{Break}{break}
\SetKw{Downto}{downto}
\SetKwProg{Fn}{Function}{:}{end}
\SetKwFunction{KeyGen}{KeyGen}


%---------------------------My Preamble
\usepackage{marvosym} %Lightning
\usepackage{booktabs}
\usepackage{multicol}
\setlength{\columnsep}{2cm}
\setlength{\columnseprule}{1.25pt}
\usepackage{enumerate}
\usepackage{soul}
\newcommand{\mathcolorbox}[2]{\colorbox{#1}{$\displaystyle #2$}}
\usepackage{graphicx}
\usepackage{tikz}
\usepackage{tikz-cd}
\usetikzlibrary{calc}
\usetikzlibrary{arrows, arrows.meta, positioning, shapes.multipart}

%Tcolorbox
\usepackage[most]{tcolorbox}
\tcbset{colback=white, arc=5pt}
%\tcbset{enhanced, colback=white,colframe=black,fonttitle=\bfseries,arc=4mm,boxrule=1pt,shadow={2mm}{-1mm}{0mm}{black!50}}
%White box with black text and shadow
%\begin{tcolorbox}[colback=white,colframe=black,fonttitle=\bfseries,title=Black Shadow Box,arc=4mm,boxrule=1pt,shadow={2mm}{-1mm}{0mm}{black!50}]
%	This is a white box with black text and a subtle shadow. The shadow adds some depth and dimension to the box without overpowering the design.
%\end{tcolorbox}

%Theorem
\newtheorem{axiom}{Axiom}[chapter]
\newtheorem{theorem}{Theorem}[chapter]
\newtheorem{proposition}[theorem]{Proposition}
\newtheorem{corollary}{Corollary}[theorem]
\newtheorem{lemma}[theorem]{Lemma}

\theoremstyle{definition}
\newtheorem{definition}{Definition}[chapter]
\newtheorem{remark}{Remark}[chapter]
\newtheorem{exercise}{Exercise}[chapter]
\newtheorem{example}{Example}[chapter]
\newtheorem*{note}{Note}

%New Command
\newcommand{\N}{\mathbb{N}}
\newcommand{\Z}{\mathbb{Z}}
\newcommand{\Q}{\mathbb{Q}}
\newcommand{\R}{\mathbb{R}}
\newcommand{\C}{\mathbb{C}}
\newcommand{\F}{\mathbb{F}}

\newcommand{\ie}{\textnormal{i.e.}}
\newcommand{\eg}{\textnormal{e.g.}}

\newcommand{\of}[1]{\left( #1 \right)} 

\newcommand{\nbhd}{\mathcal{N}}
\newcommand{\Id}{\operatorname{\textnormal{id}}}

%\newcommand{\norm}[1]{\left\| #1 \right\|}

\newcommand{\sol}{\textcolor{magenta}{\bf Sol}}

\newcommand{\inv}[1]{{#1}^{-1}}
\newcommand{\img}{\textnormal{Im}}

\newcommand{\by}{\times}
\newcommand{\Span}[1]{\textnormal{span}\langle #1\rangle}
\newcommand{\Sspan}[1]{\textnormal{span}\bigg\langle #1\bigg\rangle}
\newcommand{\basis}{\mathscr{B}}
\newcommand{\scrC}{\mathscr{C}}
\newcommand{\rank}{\textnormal{rank}}

% Begin document
\begin{document}
	
	% Title page
	\begin{titlepage}
		\begin{center}
			{\Huge\textsf{\textbf{Advanced Applied Mathematics}}\par}
			\vspace{0.5in}
			{\Large Ji Yong-Hyeon\par}
			\vspace{1in}
			\includegraphics[scale=.5]{aam.png}\par
			\vspace{1in}
			{Department of Information Security, Cryptology, and Mathematics\par}
			{College of Science and Technology\par}
			{Kookmin University\par}
			%\includegraphics[width=1.5in]{school_logo.jpg}\par
			\vspace{.25in}
			{\large \today\par}
		\end{center}
	\end{titlepage}
	
	% Table of contents
	\tableofcontents
	
	% Chapters
	\mainmatter
	
	\chapter{Linear Algebra}
	\section{System of Linear Equations}
	\begin{itemize}
		\item A system of linear equations
		\[
		\begin{cases}
			x_1,\dots,x_n:\text{unknowns}\\
			\text{\# of unknowns}=n\\
			\text{\# of equations}=m
		\end{cases}
		\]
		\begin{align*}
			&\begin{cases}
				a_{11}x_1+a_{12}x_2+\cdots+a_{1n}x_n=b_1\\
				a_{21}x_1+a_{22}x_2+\cdots+a_{2n}x_n=b_2\\
				\hspace{50pt}\vdots\\
				a_{m1}x_1+a_{m2}x_2+\cdots+a_{mn}x_n=b_m\\
			\end{cases}\\
			\iff&\begin{bmatrix}
				a_{11} & a_{12} & \cdots & a_{1n}\\
				\vdots \\
				a_{m1} & a_{m2} &\cdots & a_{mn}
			\end{bmatrix}\begin{bmatrix}
				x_1\\ \vdots \\ x_n
			\end{bmatrix}=\begin{bmatrix}
				b_1\\ \vdots \\ b_m
			\end{bmatrix} &\textcolor{blue}{A\textbf{x}=\textbf{b}}\\
			\iff& x_1\begin{bmatrix}
				a_{11}\\ a_{21}\\ \vdots\\ a_{m1}
			\end{bmatrix}+x_2\begin{bmatrix}
				a_{12}\\ a_{22}\\ \vdots\\ a_{m2}
			\end{bmatrix}+\cdots+x_n\begin{bmatrix}
				a_{1m}\\ a_{2m}\\ \vdots\\ a_{mn}
			\end{bmatrix}=\begin{bmatrix}
				b_1\\ b_2\\ \vdots\\ b_{m} 
			\end{bmatrix} & \textcolor{blue}{x_1\textbf{C}_1+\cdots+x_n\textbf{C}_n=\textbf{b}}
		\end{align*}
		\item Matrix operation
		\begin{enumerate}[(i)]
			\item scalar multiplication: $kA$
			\item addition: $A+B$
			\item multiplication: $AB$
		\end{enumerate}
		\item Properties \begin{itemize}
			\item Associative: $(A+B)+C=A+(B+C)$, $A(BC)=(AB)C$
			\item Distributive: $(AB)C=A(BC)$
			\item (in general) not commutative: $AB\neq BA$
		\end{itemize}
		\item Transpose of $A$: $A^T$ \[
		(a_{ij})_{m\times n}\longrightarrow(a^t_{ij})_{n\times m}=(a_{ji})_{n\times m}
		\]
		\item Square Matrices
	\end{itemize}
	
	\subsection{Elementary Transformation}
	\begin{itemize}
		\item Exchange of two equations (rows in the matrix representing the system
		of equations)
		\item Multiplication of an equation (row) with a constant $\lambda\in\R^*$
		\item Addition of two equations (rows)
	\end{itemize}
	
	\begin{remark}
		$A\textbf{x}=\textbf{b}\iff[A\mid\textbf{b}]$.
	\end{remark}
	
	\begin{example}
		\begin{align*}
			\begin{bmatrix}
				1 & 1 & 1\\ 1 & -1& 2\\ 2 & 0 & 3
			\end{bmatrix}\begin{bmatrix}
				x_1\\ x_2\\ x_3
			\end{bmatrix}=\begin{bmatrix}
				3 \\ 2 \\ 5
			\end{bmatrix} &\iff \left[
			\begin{array}{ccc|c}
				1 & 1 & 1 & 3\\
				1 & -1 & 2 & 2\\
				2 & 0 & 3 & 5
			\end{array}
			\right] \\ &\xLeftrightarrow[R_3\gets R_3-2R_1]{R_2\gets R_2-R_1} \left[
			\begin{array}{ccc|c}
				1 & 1 & 1 & 3\\
				0 & -2 & 1 & -1\\
				0 & -2 & 1 & -1
			\end{array} 
			\right] \\
			&\xLeftrightarrow[R_2\gets -\frac{1}{2}R_2]{R_3\gets R_3-R_2} \left[
			\begin{array}{ccc|c}
				1 & 1 & 1 & 3\\
				0 & 1 & -1/2 & 1/2\\
				0 & 0 & 0 & 0
			\end{array} 
			\right]\quad\text{Row-Echelon Form (REF)} \\
			&\xLeftrightarrow{R_1\gets R_1-R_2} \left[
			\begin{array}{ccc|c}
				1 & 0 & 3/2 & 5/2\\
				0 & 1 & -1/2 & 1/2\\
				0 & 0 & 0 & 0
			\end{array} 
			\right]\quad\text{Reduced Row-Echelon Form (RREF)} \\
			&\iff \begin{cases}
				x_1=-\frac{3}{2}x_3+\frac{5}{2}\\
				x_2=\frac{1}{2}x_3+\frac{1}{2}
			\end{cases}.
		\end{align*} Let $x_3=\lambda$ then \[
		\textbf{x}=\begin{bmatrix}
			-\frac{3}{2}\lambda+\frac{5}{2}\\ \frac{1}{2}\lambda+\frac{1}{2}\\ \lambda
		\end{bmatrix}=\begin{bmatrix}
		\frac{5}{2}\\ \frac{1}{2}\\ 0
		\end{bmatrix}+\lambda\begin{bmatrix}
			-\frac{3}{2}\\ \frac{1}{2}\\ 1
		\end{bmatrix}.
		\]
	\end{example}

	\section{Reduced Row-Echelon Form (RREF)}
	\section{General Solution and Minus-1 Trick}
	
	\section{Vector Space, Linear Transformation, Basis}
	
	\newpage
	\section{Matrix Representation of Linear Mappings}
	
	\section{Basis and Rank}
	\section{Linear Mappings}
	\subsection{Basis Change}
	
	\begin{tcolorbox}[colframe=thmcolor,title={\color{white}\bf Basis Change}]
		\begin{theorem}
			For a linear mapping $\Phi:V\to W$, ordered bases \[
			\mathscr{B}=(\textbf{b}_1,\dots,\textbf{b}_n),\quad\tilde{\mathscr{B}}=(\tilde{\textbf{b}}_1,\cdots\tilde{\textbf{b}}_n)
			\] of \(V\) and \[
			\mathscr{C}=(\textbf{c}_1,\dots,\textbf{c}_m),\quad\tilde{\mathscr{C}}=(\tilde{\textbf{c}}_1,\cdots\tilde{\textbf{c}}_m)
			\] of \(W\), and a transformation matrix \(\textbf{A}_{\Phi}=\sbr[1]{a_{ij}}_{m\by n}\) w.r.t. \(\mathscr{B}\) and \(\mathscr{C}\), the corresponding transformation matrix \(\tilde{\textbf{A}}_\Phi=\sbr[1]{\tilde{a}_{ij}}_{m\by n}\) w.r.t. the bases \(\tilde{\mathscr{B}}\) and \(\tilde{\mathscr{C}}\) is given \[
			\boxed{\tilde{\textbf{A}}_\Phi=\textbf{T}^{-1}\textbf{A}_\Phi \textbf{S}}.
			\]
			\begin{tikzcd}
				&& V \arrow[rr, "\Phi"] && W && V \arrow[rr, "\Phi"] && W\\
				&& \mathscr{B} \arrow[rr, "\textbf{A}_\Phi"] && \mathscr{C} && \mathscr{B} \arrow[rr, "\textbf{A}_\Phi"] && \mathscr{C} \arrow[dd, "\textbf{T}^{-1}"] \\
				&& && && &&\\
				&& \tilde{\mathscr{B}} \arrow[uu, "\textbf{S}"] \arrow[rr, "\tilde{\textbf{A}}_\Phi"] && \tilde{\mathscr{C}} \arrow[uu, "\textbf{T}"'] && \tilde{\mathscr{B}} \arrow[uu, "\textbf{S}"] \arrow[rr, "\tilde{\textbf{A}}_\Phi"] && \tilde{\mathscr{C}}             
			\end{tikzcd}
		\end{theorem}
	\end{tcolorbox}
	\begin{proof}
		Let \[
			\textbf{S}:=\sbr[1]{s_{ij}}_{n\by n}=\sbr[2]{\tilde{\textbf{b}}_1\ \tilde{\textbf{b}}_2\ \cdots\ \tilde{\textbf{b}}_n}_{\basis},\quad
			\text{and}\quad
			\textbf{T}:=\sbr[1]{t_{lk}}_{m\by m}=\sbr[2]{\tilde{\textbf{c}}_1\ \tilde{\textbf{c}}_2\ \cdots\ \tilde{\textbf{c}}_m}_{\scrC}.
		\] That is, \[
		\tilde{\textbf{b}}_j=\begin{bmatrix}
			s_{1j} \\ \vdots \\ s_{nj}
		\end{bmatrix}_{\mathscr{B}}=\sum_{i=1}^ns_{ij}\textbf{b}_j\quad\text{and}\quad \tilde{\textbf{c}}_k=\begin{bmatrix}
		t_{1k} \\ \vdots \\ t_{mk}
		\end{bmatrix}_{\mathscr{C}}=\sum_{l=1}^mt_{lk}\textbf{c}_l
		\] for $j=1,\dots,n$ and $k=1,\dots,m$, respectively. We must show that \[
		\textbf{T}\tilde{\textbf{A}_\Phi}=\textbf{A}_\Phi\textbf{S}\in M_{m\by n}(\R).
		\] \begin{enumerate}[(i)]
			\item \((\textbf{T}\tilde{\textbf{A}_\Phi})\) For \(j=1,2,\dots,n\), \[
			\Phi(\tilde{\textbf{b}}_j)=\sum_{k=1}^{m}\tilde{a}_{kj}\tilde{\textbf{c}}_k=\sum_{k=1}^m\sbr{\tilde{a}_{kj}\del{\sum_{l=1}^mt_{lk}\textbf{c}_l}}=\sum_{l=1}^m\sbr{\del{\sum_{k=1}^mt_{lk}\tilde{a}_{kj}}\textbf{c}_l}.
			\]
			\item \((\textbf{A}_\Phi\textbf{S})\) For \(j=1,2,\dots,n\), \[
			\Phi(\tilde{\textbf{b}}_j)=\Phi\of{\sum_{i=1}^ns_{ij}\textbf{b}_j}=\sum_{i=1}^n\sbr{s_{ij}\Phi(\textbf{b}_i)}=\sum_{i=1}^n\sbr{s_{ij}\sum_{i=1}^ma_{li}\textbf{c}_l}=\sum_{l=1}^m\of{\sum_{i=1}^na_{li}s_{ij}}\textbf{c}_l.
			\]
		\end{enumerate}
		 Hence \[
		 \sum_{k=1}^mt_{lk}\tilde{a}_{kj}=\sum_{i=1}^na_{li}s_{ij}\implies\textbf{T}\tilde{\textbf{A}_\Phi}=\textbf{A}_\Phi\textbf{S}\implies\tilde{\textbf{A}}_\Phi=\textbf{T}^{-1}\textbf{A}_\Phi \textbf{S}.
		 \]
	\end{proof}
	
	\begin{example}
		Let \[
		y_1\textbf{e}_1+y_2\textbf{e}_2=\Phi(x_1\textbf{e}_1+x_2\textbf{e}_2)=(x_1+5x_2)\textbf{e}_1+6x_2\textbf{e}_2.
		\] Then \[\begin{bmatrix}
			y_1\\ y_2
		\end{bmatrix}=\textbf{A}_\Phi\begin{bmatrix}
	x_1 \\ x_2
\end{bmatrix},\quad\text{where}\quad
\textbf{A}_\Phi=\begin{bmatrix}
	1 & 5\\ 0 & 6
\end{bmatrix}.
		\] We define \[
		\tilde{\basis}=\sbr{\tilde{\textbf{b}}_1\ \tilde{\textbf{b}}_2}:=\begin{bmatrix}
			1 & 1\\ 1 & 0
		\end{bmatrix}.
		\]\[
		\tilde{\textbf{A}}_\Phi=\textbf{T}^{-1}\textbf{A}_\Phi\textbf{S}=
		\begin{bmatrix}
			0 & 1\\ 1 & -1
		\end{bmatrix}\begin{bmatrix}
		0 & 5\\ 0 & 6
	\end{bmatrix}\begin{bmatrix}
	1 & 1\\ 1 & 0
\end{bmatrix}=\begin{bmatrix}
6 & 0\\ 0 & 1
\end{bmatrix}
		\] \[
		\Phi\of{\tilde{x}_1\begin{bmatrix}
			1 \\ 1
		\end{bmatrix}+\tilde{x}_2\begin{bmatrix}
		1 \\ 0
	\end{bmatrix}}=6\tilde{x}_1\begin{bmatrix}
1 \\ 1
\end{bmatrix}+\tilde{x}_2\begin{bmatrix}
1 \\ 0
\end{bmatrix}.
		\]
	\end{example}

	\begin{tcolorbox}[colframe=defcolor,title={\color{white}\bf Similarity}]
		\begin{definition}
			Let \(\textbf{A},\tilde{\textbf{A}}\in M_{n\by n}(\R)\). \(\textbf{A},\tilde{\textbf{A}}\) are \textbf{similar} if \[
			\exists\textbf{S}\in M_{n\by n}(\R):\tilde{\textbf{A}}=\textbf{S}^{-1}\textbf{A}\textbf{S}.
			\]
		\end{definition}
	\end{tcolorbox}
	
	\newpage
	\subsection{Image and Kernel}
	\begin{tcolorbox}[colframe=defcolor,title={\color{white}\bf Image and Kernel}]
		\begin{definition}
			Let \(\Phi:V\to W\) be a linear mapping. \begin{enumerate}[(1)]
				\item The \textbf{kernel (null) space} is defined by \[
				\ker(\Phi):=\Phi^{-1}(\textbf{0}_W)=\set{\textbf{v}\in V:\Phi(\textbf{v})=\textbf{0}_W}.
				\]
				\item The \textbf{image (range)} is defined by \[
				\img(\Phi):=\Phi[V]=\set{\textbf{w}\in W:(\exists\textbf{v}\in V)\ \Phi(\textbf{v})=\textbf{w}}.
				\]
			\end{enumerate}
		\end{definition}
	\end{tcolorbox}
	\begin{remark}
		\ \begin{enumerate}[(1)]
			\item \(\textbf{0}_V\in\ker(\Phi)\implies\ker\Phi\neq\emptyset\).
			\item \(\ker(\Phi)\subseteq V\) is a subspace of $V$.
			\item \(\img(\Phi)\subseteq W\) is a subspace of $W$.
			\item \(\Phi:V\rightarrowtail W\iff \ker(\Phi)=\set{\textbf{0}_V}\).
		\end{enumerate}
	\end{remark}
	\vspace{8pt}
	\begin{remark}[Null Space and Column Space]
		Let \(\textbf{A}\in M_{m\by n}(\R)\) and $$\fullfunction{\Phi}{\R^n}{\R^m}{\textbf{x}}{\textbf{Ax}}$$
		\begin{enumerate}[(1)]
			\item The \textbf{column space} is the image of \(\Phi\), the span of the columns of \(\textbf{A}\),\begin{align*}
				\img(\Phi)=\set{\textbf{A}\textbf{x}:\textbf{x}\in\R^n}&=\set{\sbr{\textbf{a}_1,\dots,\textbf{a}_n}\begin{bmatrix}
						x_1\\ \vdots\\ x_n
					\end{bmatrix}:x_i\in\R}\\
				&=\set{\sum_{i=1}^nx_i\textbf{a}_i:x_i\in\R}\\
				&=\Span{\textbf{a}_1,\dots,\textbf{a}_n}\subseteq\R^m.
			\end{align*}
			\item \(\rank(\textbf{A})=\dim(\img(\Phi))\).
			\item The \textbf{null space} \(\ker(\Phi)\) is $\set{\textbf{x}:\textbf{Ax}=\textbf{0}}$.
		\end{enumerate}
	\end{remark}

	\newpage
	\begin{example}[Image and Kernel of Linear Mapping]
		The mapping \begin{align*}
			\Phi:\R^4\to\R^2:\begin{bmatrix}
				x_1\\x_2\\x_3\\x_4
			\end{bmatrix}\mapsto\begin{bmatrix}
			1&2&-1&0\\1&0&0&1
		\end{bmatrix}\begin{bmatrix}
			x_1\\x_2\\x_3\\x_4
		\end{bmatrix}&=\begin{bmatrix}
			x_1+2x_2-x_3\\x_1+x_4
		\end{bmatrix}\\
		&=x_1\begin{bmatrix} 1\\1 \end{bmatrix}+x_2\begin{bmatrix} 2\\0 \end{bmatrix}+x_3\begin{bmatrix} -1\\0 \end{bmatrix}+x_4\begin{bmatrix} 0\\1 \end{bmatrix}.
		\end{align*} is linear. Then \begin{enumerate}[(1)]
		\item \(\img(\Phi)=\textnormal{span}\bigg\langle
			\begin{bmatrix}1\\1\end{bmatrix},
			\begin{bmatrix}2\\0\end{bmatrix},
			\begin{bmatrix}-1\\0\end{bmatrix},
			\begin{bmatrix}0\\1\end{bmatrix}
			\bigg\rangle=\R^2
			\)
			\item Since \[
			\begin{bmatrix}
				1&2&-1&0\\1&0&0&1
			\end{bmatrix}\rightsquigarrow\cdots\rightsquigarrow\begin{bmatrix}
			1&0&0&1\\0&1&-\frac{1}{2}&-\frac{1}{2}
		\end{bmatrix}\xrightarrow[]{\text{Minus-1 Trick}}
			\begin{bmatrix}
				1&0&0&1\\0&1&-\frac{1}{2}&-\frac{1}{2}\\
				0&0&-1&0\\ 0&0&0&-1
			\end{bmatrix},
			\] we have \[
			\ker(\Phi)=\Sspan{\begin{bmatrix}1\\-1/2\\-1\\0\end{bmatrix},
					\begin{bmatrix}1\\-1/2\\0\\-1\end{bmatrix}}.
			\]
	\end{enumerate}
	\end{example}
	\vspace{8pt}
	\begin{tcolorbox}[colframe=thmcolor,title={\color{white}\bf Rank-Nullity Theorem (Fundamental Theorem of Linear Mapping)}]
		\begin{theorem}
			Let \(\Phi:V\to W\) be a linear mapping for vector spaces \(V,W\). Then \[
			\dim(\ker\Phi)+\dim(\img\Phi)=\dim V.
			\]
		\end{theorem}
	\end{tcolorbox}

	\section{Affine Spaces}
	\[
	\Phi(\textbf{x})=\textbf{A}\textbf{x}+\textbf{b}
	\] \(\img\Phi\) is not a subspace if \(\textbf{b}\neq 0\).
	
	\newpage
	\chapter{Analytic Geometry}
	
	\section{Norm}
	\begin{tcolorbox}[colframe=defcolor,title={\color{white}\bf Norm}]
		\begin{definition}
			A \textbf{norm} on a vector space \(V\) is a function \[
			\fullfunction{\norm{\ \cdot\ }}{V}{\R}{\textbf{x}}{\norm{\textbf{x}}}
			\] such that for all \(\lambda\in\R\) and \(\textbf{x},\textbf{y}\in V\) the following hold: \begin{enumerate}[(i)]
				\item (Absolutely homogeneous)\quad \(\norm{\lambda x}=\abs{\lambda}\norm{\textbf{x}}\)
				\item (Triangle inequality)\quad \(\norm{\textbf{x}+\textbf{y}}\leq\norm{\textbf{x}}+\norm{\textbf{y}}\)
				\item (Positive definite)\quad \(\norm{\textbf{x}}\) and \(\norm{x}=0\Leftrightarrow\textbf{x}=\textbf{0}\).
			\end{enumerate}
		\end{definition}
	\end{tcolorbox}
	\vspace{8pt}
	\begin{example}[Manhattan Norm]
		The Manhattan norm on $\R^n$ is defined for $\textbf{x}\in\R^n$ as $$\norm{\textbf{x}}_1:=\sum_{i=1}^n\abs{x_i}.$$ The Manhattan norm is also called \(\mathscr{l}_1\) norm.
	\end{example}
	\begin{example}[Euclidean Norm]
		The Manhattan norm on $\R^n$ is defined for $\textbf{x}\in\R^n$ as $$\norm{\textbf{x}}_2:=\sqrt{\sum_{i=1}^nx_i^2}=\sqrt{\textbf{x}^T\textbf{x}}.$$ The Euclidean norm is also called \(\mathscr{l}_2\) norm.
	\end{example}

	\section{Inner Products}
	\subsection{General Inner Product}
	\begin{tcolorbox}[colframe=defcolor,title={\color{white}\bf Dot Product (Scalar Product)}]
		\begin{definition}
			The \textbf{dot product (scalar product)} in \(\R^n\) is given by \[
			\textbf{x}\cdot\textbf{y}=\textbf{x}^T\textbf{y}=\sum_{i=1}^nx_iy_i.
			\]
		\end{definition}
	\end{tcolorbox}
	\vspace{8pt}
	\begin{tcolorbox}[colframe=defcolor,title={\color{white}\bf Bilinaer Mapping}]
		\begin{definition}
			Let \(V\) be a vector space and \(\Omega:V\times V\to\R\) is a \textbf{bilienar mapping} if for all \(\alpha,\beta\in\R\),  \begin{enumerate}[(i)]
				\item $\Omega(\alpha \textbf{x}_1+\beta\textbf{x}_2,\textbf{y}) = \alpha\Omega(\textbf{x},\textbf{y})+\beta\Omega(\textbf{x}_2,y)$.
				\item $\Omega(\textbf{y},\alpha\textbf{y}_1+\beta\textbf{y}_2) = \alpha\Omega(\textbf{x},\textbf{y}_1)+\beta\Omega(\textbf{x},y_2)$.
			\end{enumerate}
		\end{definition}
	\end{tcolorbox}
	\begin{remark}
		\ \begin{enumerate}[(1)]
			\item \(\Omega\) is called \textbf{symmetric} if \(\forall\textbf{x},\textbf{y}\in V:\Omega(\textbf{x},\textbf{y})=\Omega(\textbf{y},\textbf{x})\).
			\item \(\Omega\) is called \textbf{positive definite} if $
			\forall \textbf{x}\in V\setminus\set{\textbf{0}}:\Omega(\textbf{x},\textbf{x})>0,\ \Omega(\textbf{0},\textbf{0})=0.$
		\end{enumerate}
	\end{remark}
	\vspace{8pt}
	\begin{tcolorbox}[colframe=defcolor,title={\color{white}\bf Inner Product}]
		\begin{definition}
			A bilinear mapping \(V\times V\to\R\) is called an \textbf{inner product} on \(V\) if \(\Omega\) is symmetric and positive definite bilinear mapping. 
		\end{definition}
	\end{tcolorbox}
	
	\newpage
	\section{Positive Definite, Gram-Schmidt Process}
	
	\begin{tcolorbox}[colframe=defcolor,title={\color{white}\bf }]
		\begin{definition}
			
		\end{definition}
	\end{tcolorbox}
	\begin{tcolorbox}[colframe=thmcolor,title={\color{white}\bf }]
		\begin{theorem}
			
		\end{theorem}
	\end{tcolorbox}
	
	\begin{tcolorbox}[colframe=defcolor,title={\color{white}\bf }]
		\begin{definition}
			
		\end{definition}
	\end{tcolorbox}
	\begin{tcolorbox}[colframe=thmcolor,title={\color{white}\bf }]
		\begin{theorem}
			
		\end{theorem}
	\end{tcolorbox}

	\begin{tcolorbox}[colframe=defcolor,title={\color{white}\bf }]
		\begin{definition}
			
		\end{definition}
	\end{tcolorbox}
	\begin{tcolorbox}[colframe=thmcolor,title={\color{white}\bf }]
		\begin{theorem}
			
		\end{theorem}
	\end{tcolorbox}
	
	\chapter{Matrix Decompositions}
	\section{Eigenvalues and Eigenvectors}
	\section{Complex Matrix, Hermitian}
	\section{Decomposition of Matrices: Spectral Decomposition, SVD}
	
	\chapter{Vector Calculus}
	\section{Vector Calculus}
	
	\chapter{Probability and Distributions}
	\section{Backpropagation, Probability Theory Basic}
	\section{Gaussian Distribution and its Applications}
	\section{Continuous Optimization}
	\section{Probabilistic Modeling and Inference}
	\section{Linear Regression}
	\section{Supervised Learning}
	\section{Bayesian Linear Regression}
	\section{Maximum Likelihood Estimation(MLE)}
	\section{Generalized Linear Regression}
	\section{Maximum A Posteriori Estimation(MAP)}
	\section{Principal Component Analysis(PCA)}
	
	\chapter{Continuous Optimization}
	
	\chapter{When Models Meet Data}
	
	\chapter{Linear Regression}
	
	\chapter{Dimensionality Reduction with PCA}
	% End document
\end{document}
