\documentclass[12pt,openany]{book}

\usepackage{amsmath,amsthm,amsfonts,amscd} % Packages for mathematics
\usepackage{commath}

% Colors
\usepackage{xcolor}
\definecolor{titleblue}{RGB}{0,53,128}
\definecolor{chaptergray}{RGB}{140,140,140}
\definecolor{sectiongray}{RGB}{180,180,180}

\definecolor{thmcolor}{RGB}{231, 76, 60}
\definecolor{defcolor}{RGB}{52, 152, 219}
\definecolor{lemcolor}{RGB}{155, 89, 182}
\definecolor{corcolor}{RGB}{46, 204, 113}
\definecolor{procolor}{RGB}{241, 196, 15}

% Fonts
\usepackage[T1]{fontenc}
\usepackage[utf8]{inputenc}
\usepackage{newpxtext,newpxmath}
\usepackage{sectsty}
\allsectionsfont{\sffamily\color{titleblue}\mdseries}

% Page layout
\usepackage{geometry}
\geometry{a4paper,left=1.325in,right=1in,top=1in,bottom=1in,heightrounded}
\usepackage{fancyhdr}
\fancyhf{}
\fancyhead[LE,RO]{\thepage}
\fancyhead[LO]{\nouppercase{\rightmark}}
\fancyhead[RE]{\nouppercase{\leftmark}}
\renewcommand{\headrulewidth}{0.5pt}
\renewcommand{\footrulewidth}{0pt}

% Chapter formatting
\usepackage{titlesec}
\titleformat{\part}[display]
{\normalfont\sffamily\Huge\bfseries\color{titleblue!80!black}\filcenter}
{\partname\ \thepart}{20pt}{\Huge}
\titleformat{\chapter}[display]
{\normalfont\sffamily\Huge\bfseries\color{titleblue}}{\chaptertitlename\ \thechapter}{20pt}{\huge}
\titleformat{\section}
{\normalfont\sffamily\Large\bfseries\color{titleblue!100!gray}}{\thesection}{1em}{}
\titleformat{\subsection}
{\normalfont\sffamily\large\bfseries\color{titleblue!75!gray}}{\thesubsection}{1em}{}

% Table of contents formatting
\usepackage{tocloft}
\renewcommand{\cftchapfont}{\sffamily\color{titleblue}\bfseries}
\renewcommand{\cftsecfont}{\sffamily\color{chaptergray}}
\renewcommand{\cftsubsecfont}{\sffamily\color{sectiongray}}
\renewcommand{\cftchapleader}{\cftdotfill{\cftdotsep}}

% Hyperlinks
\usepackage{hyperref}
\hypersetup{
	colorlinks=true,
	linkcolor=titleblue,
	filecolor=black,      
	urlcolor=titleblue,
}

%Listing
\usepackage{listings} %Code
\renewcommand{\lstlistingname}{Code}%

\definecolor{sagegreen}{rgb}{0.0,0.6,0.4}
\definecolor{sagepurple}{rgb}{0.6,0.0,0.4}
\definecolor{sageblue}{rgb}{0.0,0.4,0.6}
\definecolor{sageorange}{rgb}{1.0,0.4,0.0}
\definecolor{sagegray}{rgb}{0.4,0.4,0.4}

\lstdefinestyle{sage}{
	language=Python,
	backgroundcolor=\color{white},
	basicstyle=\small\ttfamily\color{black}, 
	basicstyle=\footnotesize\ttfamily\color{black},
	keywordstyle=\color{blue!60!black},
	commentstyle=\color{green!60!black},
	stringstyle=\color{purple!60!black},
	showstringspaces=false,
	breaklines=true,
	tabsize=4,
	morekeywords={True, False, None},
	frame=leftline, % Remove the border
	framesep=3pt,
	frameround=tttt,
	framexleftmargin=3pt,
	numbers=left,
	numberstyle=\small\color{gray},
	xleftmargin=15pt, % Increase the left margin
	xrightmargin=5pt,
	captionpos=b,
	belowskip=0pt,
	aboveskip=4pt
}

\lstdefinestyle{C}{
	language=C,
	basicstyle=\ttfamily\footnotesize,
	backgroundcolor=\color{white},
	keywordstyle=\color{purple}\bfseries,
	stringstyle=\color{orange},
	commentstyle=\color{green!70!black}\itshape,
	directivestyle=\color{blue},
	numberstyle=\tiny\color{gray},
	numbers=left,
	numbersep=8pt, % Increase separation between line numbers and code
	breaklines=true,
	breakatwhitespace=true,
	postbreak=\mbox{\textcolor{red}{$\hookrightarrow$}\space},
	tabsize=4,
	frame=single,
	framerule=0.5pt, % Thin frame to mimic typical PDF borders
	rulecolor=\color{gray!60},
	title=\lstname, 
	keywordstyle=[2]{\color{magenta}},
	keywords=[2]{printf},
	emph={int,char,double,float,unsigned},
	emphstyle={\color{blue}},
	escapeinside={(*@}{@*)},
	morekeywords={*,...},
	deletekeywords={...}, % if you want to delete keywords from the default language
	showspaces=false, % Show spaces as visible characters
	showstringspaces=false, % But not within strings to mimic common PDF copy artifacts
	columns=flexible, % Mimic irregular spacing found when copying from PDFs
}

%Ceiling and Floor Function
\usepackage{mathtools}
\DeclarePairedDelimiter{\ceil}{\lceil}{\rceil}
\DeclarePairedDelimiter{\floor}{\lfloor}{\rfloor}

%Algorithm
\usepackage[ruled,linesnumbered]{algorithm2e}
\usepackage{setspace}
\usepackage{algpseudocode}
\SetKwComment{Comment}{/* }{ */}
\SetKw{Break}{break}
\SetKw{End}{end}
\SetKw{Downto}{downto}
\SetKwProg{Fn}{Function}{:}{end}
\SetKwProg{Procedure}{procedure}{:}{end}
\SetKwFunction{KeyGen}{KeyGen}

%---------------------------My Preamble
\usepackage{marvosym} %Lightning
\usepackage{booktabs}
\usepackage{multicol}
\setlength{\columnsep}{2cm}
\setlength{\columnseprule}{1.25pt}
\usepackage{enumerate}
\usepackage{soul}
\newcommand{\mathcolorbox}[2]{\colorbox{#1}{$\displaystyle #2$}}
\usepackage{graphicx}
\usepackage{tikz}
\usepackage{tikz-cd}
\usepackage{circuitikz}
\usetikzlibrary{calc}
\usetikzlibrary{arrows, arrows.meta, positioning, shapes.multipart}
\usepackage{pgfplots}
\pgfplotsset{compat=1.16}
\usepackage{caption}

%Tcolorbox
\usepackage[most]{tcolorbox}
\tcbset{colback=white, arc=5pt}
%\tcbset{enhanced, colback=white,colframe=black,fonttitle=\bfseries,arc=4mm,boxrule=1pt,shadow={2mm}{-1mm}{0mm}{black!50}}
%White box with black text and shadow
%\begin{tcolorbox}[colback=white,colframe=black,fonttitle=\bfseries,title=Black Shadow Box,arc=4mm,boxrule=1pt,shadow={2mm}{-1mm}{0mm}{black!50}]
%	This is a white box with black text and a subtle shadow. The shadow adds some depth and dimension to the box without overpowering the design.
%\end{tcolorbox}

%Theorem
\newtheorem{axiom}{Axiom}[chapter]
\newtheorem{theorem}{Theorem}[chapter]
\newtheorem{proposition}[theorem]{Proposition}
\newtheorem{corollary}{Corollary}[theorem]
\newtheorem{lemma}[theorem]{Lemma}

\theoremstyle{definition}
\newtheorem{definition}{Definition}[chapter]
\newtheorem{remark}{Remark}[chapter]
\newtheorem{exercise}{Exercise}[chapter]
\newtheorem{example}{Example}[chapter]
\newtheorem*{note}{Note}

%New Command
\newcommand{\N}{\mathbb{N}}
\newcommand{\Z}{\mathbb{Z}}
\newcommand{\Q}{\mathbb{Q}}
\newcommand{\R}{\mathbb{R}}
\newcommand{\C}{\mathbb{C}}
\newcommand{\F}{\mathbb{F}}
 
\renewcommand{\abs}[1]{\left\lvert #1 \right\rvert}
\renewcommand{\norm}[1]{\left\| #1 \right\|}

\newcommand{\abss}[1]{\lvert #1 \rvert}

\newcommand{\ie}{\textnormal{i.e.}}
\newcommand{\eg}{\textnormal{e.g.}}

\newcommand{\of}[1]{\left( #1 \right)} 

\newcommand{\nbhd}{\mathcal{N}}
\newcommand{\Id}{\operatorname{\textnormal{id}}}

%\newcommand{\norm}[1]{\left\| #1 \right\|}

\newcommand{\sol}{\textcolor{magenta}{\bf Solution}}

\newcommand{\inv}[1]{{#1}^{-1}}
\newcommand{\img}{\textnormal{Im}}

\newcommand{\tab}{\hspace{8pt}}

% Begin document
\begin{document}
	
	% Title page
	\begin{titlepage}
		\begin{center}
			{\Huge\textsf{\textbf{Algorithm Analysis}}\par}
			\vspace{0.5in}
			{\Large Ji Yong-Hyeon\par}
			\vspace{1in}
			\includegraphics[scale=2]{aa2.jpg}\par
			\vspace{1in}
			{\bf Department of Information Security, Cryptology, and Mathematics\par}
			{College of Science and Technology\par}
			{Kookmin University\par}
			%\includegraphics[width=1.5in]{school_logo.jpg}\par
			\vspace{.25in}
			{\large \today\par}
		\end{center}
	\end{titlepage}
	
	% Table of contents
	\tableofcontents
	
	% Chapters
	\mainmatter
	
	\part{Basic Algorithms}
	
	\chapter{Basic Algorithms}
	\section{Rotation Algorithm}
	\begin{tcolorbox}[colback=white,colframe=lemcolor,arc=5pt,title={\color{white}\bf Rotation}]
		\textbf{Problem.}\ Rotate vector \(x[n]\) by $d$ positions.\\
		\textbf{Constraints.}\ $O(n)$ time, $O(1)$ extra space.
	\end{tcolorbox}
	\begin{example}
		For $n=8$ and $d=3$, change \[
		abcdef\to defghabc.
		\]
	\end{example}

	\newpage
	\subsection{A Juggling Algorithm}
	
	\begin{lstlisting}[style=C, caption={Juggling Rotation Algorithm},captionpos=t]
// Function to find the greatest common divisor (GCD)
int gcd(int a, int b) {
	return b ? gcd(b, a%b) : a;
}

// Function to rotate an array using Juggling Alg.
void juggling(int arr[], int n, int d) {
	int i, j, k, temp;
	
	// Find GCD of n and d
	int g = gcd(d, n);
	
	for (i = 0; i < g; i++) {
		// Store the first element of the current set
		temp = arr[i];
		
		j = i;
		
		// Shift each element of the set
		while (1) {
			k = j + d;
			
			// If k exceeds the array size, bring it within bounds
			if (k >= n) {
				k = k - n;
			}
			
			// If we are back to the initial position, break the loop
			if (k == i) {
				break;
			}
			
			// Move the element
			arr[j] = arr[k];
			
			// Update j for the next iteration
			j = k;
		}
		
		// Put the first element in its correct position within the set
		arr[j] = temp;
	}
}
	\end{lstlisting}
	
	
	\subsection{The Block-Swap Algorithm}
	
	\subsection{The Reversal Algorithm}
	
	\newpage
	\chapter{The Efficiency of Algorithms}
	
	\section{Real-valued Functions of a Real Variable\\ and Their Graphs}
	
	\begin{tcolorbox}[colframe=defcolor,title={\color{white}\bf Graph}]
		\begin{definition}
			Let $f$ be a real-valued function of a real variable. The graph of $f$ is the set \[
			\Gamma_f:=\set{(x,y)\in\R^2:y=f(x)}.
			\]
		\end{definition}
	\end{tcolorbox}
	\vspace{8pt}
	\begin{tcolorbox}[colframe=defcolor,title={\color{white}\bf Power Function}]
		\begin{definition}
			Let $a\in\R_{\geq 0}$. Define $p_a$, the power function with exponent $a$, as follows: \[
			p_a(x)=x^a\quad\text{for}\quad x\in\R_{\geq 0}.
			\]
		\end{definition}
	\end{tcolorbox}
	\begin{example}
		\ \begin{center}
			\begin{tikzpicture}
				\begin{axis}[
					title={Graphs of $x^a$ for Different $a$},
					xlabel={$x$},
					ylabel={$f(x)=x^a$},
					domain=0:1.5,
					ymin=0,
					ymax=1.5,
					samples=100,
					legend pos=north west
					]
					
					\addplot[cyan] {x^0.25};
					\addlegendentry{$a=0.25$}
					
					\addplot[blue] {x^0.5};
					\addlegendentry{$a=0.5$}
					
					\addplot[red] {x^1};
					\addlegendentry{$a=1$}
					
					\addplot[green] {x^2};
					\addlegendentry{$a=2$}
					
					\addplot[orange] {x^4};
					\addlegendentry{$a=4$}
				\end{axis}
			\end{tikzpicture}
		\end{center}
	\end{example}
	\vspace{8pt}
	\begin{tcolorbox}[colframe=defcolor,title={\color{white}\bf The Floor and Ceiling Function}]
		\begin{definition}
			\begin{align*}
				f(x)&:=\floor*{x}:=\sup\set{n\in\Z:n\leq x},\\
				g(x)&:=\ceil*{x}:=\inf\set{n\in\Z:x<n}.
			\end{align*}
		\end{definition}
	\end{tcolorbox}
	\begin{example}
		\ \begin{center}
			\begin{minipage}{.47\textwidth}
				\centering
					\begin{tikzpicture}[scale=.7,>=stealth']
						%Draw Plaid
						\draw[very thin,color=gray!15,step=1] (-4,-4) grid (5,4);
						% draw axes
						\draw[->] (-4,0) -- (5,0) node[right] {};
						\draw[->] (0,-4) -- (0,4) node[above] {};
						%Take Coordinates
						\foreach \i in {1,...,4} {
							\draw[] (\i,.1)--(\i,-.1) node[below] {$\i$};%x-axis
							\draw[] (.1,\i)--(-.1,\i) node[left] {$\i$};%y-axis
						}
						%f(x)
						\foreach \i in {-4,...,4} {
							\draw[line width=.5mm] (\i,\i) -- (\i+1,\i);
						}
						\foreach \i in {-4,...,4} {
							\fill[] (\i,\i) circle (2.5pt);
							\fill[white] (\i+1,\i) circle (2.5pt);
						}
						\foreach \i in {-4,...,4} {
							\draw[] (\i+1,\i) circle (2.5pt);
						}
					\end{tikzpicture}
					\captionof{figure}{Graph of \(f(x)=\floor{x}\).}
			\end{minipage}
			\begin{minipage}{.47\textwidth}
			\centering
			\begin{tikzpicture}[scale=.7,>=stealth']
				%Draw Plaid
				\draw[very thin,color=gray!15,step=1] (-5,-4) grid (4,4);
				% draw axes
				\draw[->] (-5,0) -- (4,0) node[right] {};
				\draw[->] (0,-4) -- (0,4) node[above] {};
				%Take Coordinates
				\foreach \i in {1,...,4} {
					\draw[] (\i,.1)--(\i,-.1) node[below] {$\i$};%x-axis
					\draw[] (.1,\i)--(-.1,\i) node[left] {$\i$};%y-axis
				}
				%f(x)
				\foreach \i in {-4,...,4} {
					\draw[line width=.5mm] (\i-1,\i) -- (\i,\i);
				}
				\foreach \i in {-4,...,4} {
					\fill[] (\i,\i) circle (2.5pt);
				}
				\foreach \i in {-5,...,3} {
					\fill[white] (\i,\i+1) circle (2.5pt);
					\draw[] (\i,\i+1) circle (2.5pt);
				}
			\end{tikzpicture}
		\captionof{figure}{Graph of \(f(x)=\ceil{x}\).}
		\end{minipage}
		\end{center}
	\end{example}
	\vspace{20pt}
	\begin{tcolorbox}[colframe=defcolor,title={\color{white}\bf Graph of a Multiple of a Function}]
		\begin{definition}
			The function \(Mf\), called the \textbf{multiple of $f$ by $M$} or \textbf{\(M\) times $f$}, is the real-valued function with the as domain as \(f\) that is defined by the rule \[
			\forall x\in\textnormal{Dom}(f):(Mf)(x)=M\cdot f(x).
			\]
		\end{definition}
	\end{tcolorbox}
	\begin{example}
		\ \begin{center}
			\begin{tikzpicture}
				\begin{axis}[
					axis lines=middle,
					xlabel={\(x\)},
					ylabel={\(y\)},
					domain=-2*pi:2*pi,
					samples=100,
					ymin=-4, ymax=4,
					xtick={-6.2832, -4.7124, -3.1416, -1.5708, 0, 1.5708, 3.1416, 4.7124, 6.2832},
					xticklabels={\( -2\pi \), \( -3\pi/2 \), \( -\pi \), \( -\pi/2 \), 0, \( \pi/2 \), \( \pi \), \( 3\pi/2 \), \( 2\pi \)},
					legend style={at={(.7,.9)},anchor=west}
					]
					% original function f(x) = sin(x)
					\addplot[blue, thick]{sin(deg(x))};
					\addlegendentry{\(f(x) = \sin(x)\)}
					
					% multiple of function g(x) = 3sin(x)
					\addplot[red, thick]{3*sin(deg(x))};
					\addlegendentry{\(g(x) = 3 \cdot \sin(x)\)}
				\end{axis}
			\end{tikzpicture}
		\end{center}
	\end{example}

	\begin{tcolorbox}[colframe=defcolor,title={\color{white}\bf Absolute Function}]
		\begin{definition}
			\[
			A(x):=\abs{x}:=\begin{cases}
				x&:x\geq0\\ -x&:x<0
			\end{cases}.
			\]
		\end{definition}
	\end{tcolorbox}
	\vspace{20pt}
	\begin{tcolorbox}[colframe=defcolor,title={\color{white}\bf Increasing and Decreasing Function}]
		\begin{definition}
			\ \begin{enumerate}[(1)]
				\item A real-valued function \(f\) is \textbf{increasing} on \(S\) iff \[
				\forall x_1,x_2\in S: x_1<x_2\implies f(x_1)<f(x_2).
				\]
				\item A real-valued function \(g\) is \textbf{decreasing} on \(S\) iff \[
				\forall x_1,x_2\in S: x_1<x_2\implies g(x_1)>g(x_2).
				\]
			\end{enumerate}
		\end{definition}
	\end{tcolorbox}

	\section{$O$, $\Sigma$, $\Theta$ Notation}
	
	\begin{note}
		\ \begin{itemize}
			\item \textbf{Algorithm} is methods for solving problems which are suited for computer implementation.
			\item \textbf{Algorithm Efficiency}
			\begin{itemize}
				\item \textbf{Time efficiency} is a measure of amount of time for an algorithm to execute.
				\item \textbf{Space efficiency} is a measure of amount of memory needed for algorithm to execute.
			\end{itemize}
			\item \textbf{$O$, $\Sigma$, $\Theta$ notation} provide approximations that make it easy to evaluate large-scale differences in algorithm efficiency, while ignoring differences of a constant factor and differences that occur only for small sets of input data.
		\end{itemize}
	\end{note}
	\vspace{20pt}
	\begin{tcolorbox}[colframe=defcolor,title={\color{white}\bf $\Omega$, \(O\) and $\Theta$}]
		\begin{definition}
			Let \(f\) and \(g\) be real-valued functions defined on the \(\R_{\geq 0}\).
			\begin{enumerate}[(1)]
				\item \textbf{$f$ is of order at least $g$}, written \(\Omega(g(x))\), if and only if, \[
				(\exists A\in\R_{> 0})(\exists a\in\R_{\geq 0})\quad x>a\implies A\abs{g(x)}\leq\abs{f(x)}.
				\]
				\item \textbf{$f$ is of order at most $g$}, written \(O(g(x))\), if and only if, \[
				(\exists B\in\R_{> 0})(\exists b\in\R_{\geq 0})\quad x>b\implies \abs{f(x)}\leq B\abs{g(x)}.
				\]
				\item \textbf{$f$ is of order $g$}, written \(\Theta(g(x))\), if and only if, \[
				(\exists A,B\in\R_{> 0})(\exists k\in\R_{\geq 0})\quad x>k\implies A\abs{g(x)}\leq\abs{f(x)}\leq B\abs{g(x)}.
				\]
			\end{enumerate}
		\end{definition}
	\end{tcolorbox}
	\vspace{20pt}
	\begin{example}[Translating to \(\Theta\)-Notation]
		Use \(\Theta\)-notation to express the statement \[
		10|x^6|\leq|17x^6-45x^3+2x+8|\leq 30|x^6|\quad\text{for all real numbers $x>2$}.
		\]
		\begin{proof}[\sol]
			Let $A=10>0$, $B=30>0$ and $k=2\geq 0$. Then \[
			A|x^6|\leq|17x^6-45x^3+2x+8|\leq B|x^6|\quad\text{for all real numbers $x>k$}.	
			\] By definition of \(\Theta\)-notation, \[
			17x^6-45x^3+2x+8\quad\text{is}\quad\Theta(x^6).
			\]
		\end{proof}
	\end{example}
	\vspace{10pt}
	\begin{example}[Translating to \(O\)- and \(\Theta\)-Notation]
		\ \begin{enumerate}[(i)]
			\item Use $\Omega$ and $O$ notations to express the statements
			\begin{enumerate}[a.]
				\item $\displaystyle 15|\sqrt{x}|\leq\abs{\frac{15\sqrt{x}(2x+9)}{x+1}}$ for all real numbers \(x>0\).
				\item $\displaystyle \abs{\frac{15\sqrt{x}(2x+9)}{x+1}}\leq45|\sqrt{x}|$ for all real numbers \(x>7\).
			\end{enumerate}
			\item Justify the statement: $\displaystyle \frac{15\sqrt{x}(2x+9)}{x+1}$ is \(\Theta(\sqrt{x})\).
		\end{enumerate}
		\newpage
		\begin{proof}[\sol]
			\ \begin{enumerate}[(i)]
				\item \begin{enumerate}[a.]
					\item Let $A=15$ and $a=0$. Then \[
					A|\sqrt{x}|\leq\abs{\frac{15\sqrt{x}(2x+9)}{x+1}}\quad\text{for all real numbers $x>a$}.	
					\] By definition of \(\Omega\)-notation, \[
					\frac{15\sqrt{x}(2x+9)}{x+1}\quad\text{is}\quad\Omega(\sqrt{x}).
					\]
					\item Let $B=45$ and $b=7$. Then \[
					\abs{\frac{15\sqrt{x}(2x+9)}{x+1}}\leq B|\sqrt{x}|\quad\text{for all real numbers $x>b$}.	
					\] By definition of \(O\)-notation, \[
					\frac{15\sqrt{x}(2x+9)}{x+1}\quad\text{is}\quad O(\sqrt{x}).
					\]
				\end{enumerate}
				\item Let $A=15,B=45$, and let $k=\max(0,7)=7$. Then \[
				A|\sqrt{x}|\leq\abs{\frac{15\sqrt{x}(2x+9)}{x+1}}\leq B|\sqrt{x}|\quad\text{for all real numbers $x>k$}.	
				\] Hence, by definition of \(O\)-notation, \[
				\frac{15\sqrt{x}(2x+9)}{x+1}\quad\text{is}\quad \Theta(\sqrt{x}).
				\]
			\end{enumerate}
		\end{proof}
	\end{example}
	
	\newpage
	\begin{tcolorbox}[colframe=thmcolor,title={\color{white}\bf }]
		\begin{theorem}
			Let \(f\) and \(g\) be real-valued functions defined on \(\R_{\geq 0}\).
			\begin{enumerate}[(1)]
				\item \(f(x)\) is \(\Omega(g(x))\) and \(f(x)\) is \(O(g(x))\) $\iff$ \(f(x)\) is \(\Theta(g(x))\).
				\item $f(x)$ is \(\Omega(g(x))\) $\iff$ $g(x)$ is $O(f(x))$.
				\item $f(x)$ is $O(g(x))$ and $g(x)$ is $O(h(x))$ $\implies$ $f(x)$ is $O(h(x))$.
			\end{enumerate}
		\end{theorem}
	\end{tcolorbox}
	\begin{proof}
		\begin{enumerate}[(1)]
			\item Clearly, it holds.
			\item \begin{itemize}
				\item[$(\Rightarrow)$] Suppose \(f(x)\) is \(\Omega(g(x))\) then \[
				(\exists A>0)(\exists a\geq0)(\forall x>a)\quad A\abss{g(x)}\leq\abss{f(x)}.
				\] Divide both sides by \(A\) to obtain \[
				(\forall x>a)\quad \abss{g(x)}\leq\frac{1}{A}\abss{f(x)}.
				\] Let $B:=1/A$ and $b:=a$. Then \[
				(\forall x>b)\quad\abss{g(x)}\leq B\abss{f(x)},
				\] and so \(g(x)\) is \(O(f(x))\) by definition of $O$-notation.
				\vspace{4pt}
				\item[$(\Leftarrow)$] Suppose \(f(x)\) is \(O(g(x))\) then \[
				(\exists B>0)(\exists b\geq0)(\forall x>b)\quad \abss{f(x)}\leq B\abss{g(x)}.
				\] Divide both sides by \(B\) to obtain \[
				(\forall x>b)\quad \frac{1}{B}\abss{f(x)}\leq\abss{g(x)}.
				\] Let $A:=1/B$ and $a:=b$. Then \[
				(\forall x>a)\quad A\abss{f(x)}\leq \abss{g(x)},
				\] and so \(g(x)\) is \(\Omega(f(x))\) by definition of $\Omega$-notation.
			\end{itemize}
			\item Suppose that $f(x)$ is $O(g(x))$ and $g(x)$ is $O(h(x))$. Then 
			\begin{align*}
				(\exists B_1,B_2\in\R_{>0})(\exists b_1,b_2\in\R_{\geq 0})\quad &x>b_1\Rightarrow\abss{f(x)}\leq B_1\abss{g(x)}\quad\text{and}\\ &x>b_2\Rightarrow\abss{g(x)}\leq B_2\abss{h(x)}.
			\end{align*}
			Let $B=B_1B_2$ and \(b=\max(b_1,b_2)\). Then \[
			x>b\implies\abss{f(x)}\leq B_1\abss{g(x)}\leq B_1(B_2|h(x)|)\leq B|h(x)|.
			\] Then, by definition of $O$-notation, $f(x)$ is $O(h(x))$.
		\end{enumerate}
	\end{proof}
	
	\newpage
	\begin{example}
		Show that \(2x^4+3x^3+5\) is $\Theta(x^4)$.
		\begin{proof}[\sol]
			Define functions $f$ and $g$ as follows: \begin{align*}
				f(x)&:=2x^4+3x^3+5,\ \text{and}\\
				g(x)&:=x^4
			\end{align*} for all $x\in\R_{\geq 0}$.
			\begin{enumerate}[(i)]
				\item For $x>0$, \begin{align*}
					2x^4&\leq 2x^4+3x^3+5\\
					2|x^4|&\leq |2x^4+3x^3+5|.
				\end{align*} Let $A=2$ and $a=0$. Then \[
				A|x^4|\leq |2x^4+3x^3+5|\quad\text{for all $x>a$},
				\] and so $2x^4+3x^3+5$ is \(\Omega(x^4)\).
				\item For $x>1$, \begin{align*}
					2x^4+3x^3+5&\leq 2x^4+3x^4+5x^4,\\
					2x^4+3x^3+5&\leq 10x^4,\\
					|2x^4+3x^3+5|&\leq 10|x^4|.
				\end{align*} Let $B=10$ and $b=1$. Then \[
			|2x^4+3x^3+5|\leq B|x^4|\quad\text{for all $x>b$},
		\] and so $2x^4+3x^3+5$ is \(O(x^4)\).
			\end{enumerate}
			By (i) and (ii), we know that $2x^4+3x^3+5$ is $\Theta(x^4)$.
		\end{proof}
	\end{example}
	\vspace{8pt}
	\begin{example}
		\ \begin{enumerate}[a.]
			\item Show that \(3x^3-1000x-200\) is $O(x^3)$.
			\item Show that \(3x^3-1000x-200\) is $O(x^s)$ for all integer $s>3$.
		\end{enumerate}
		\begin{proof}[\sol]
			\begin{enumerate}[a.]
				\item For $x>1$, \begin{align*}
					|3x^3-1000x-200|&\leq|3x^3|+|1000x|+|200|\\
					&\leq 3x^3+1000x^3+200x^3\\
					&\leq 1203x^3
				\end{align*}
			\(\therefore 3x^3-1000x-200\) is $O(x^3)$.
			\item Suppose $s$ is an integer with $s>3$. For $x>1$, we know $x^3<x^s$. Then \[
			B|x^3|< B|x^s|
			\] for $x>b$ ($\because b=1$). Thus, \[
			|3x^3-1000x-200|\leq B|x^s|\quad\text{for all $x>1$}.
			\] Hence, $3x^3-1000x-200$ is $O(x^s)$ for all integer $s>3$.
			\end{enumerate}
		\end{proof}
	\end{example}
	\newpage
	\begin{example}
		\ \begin{enumerate}[a.]
			\item Show that \(3x^3-1000x-200\) is $\Omega(x^3)$.
			\item Show that \(3x^3-1000x-200\) is $\Omega(x^r)$ for all integer $r<3$.
		\end{enumerate}
		\begin{proof}[\sol]
			\ \begin{enumerate}[a.]
				\item Let $\displaystyle a:=2\times \frac{(1000+200)}{3}=800$. Then \begin{align*}
					x&>a,\\
					x&>800,\\
					x&>\frac{2\cdot 1000}{3}+\frac{2\cdot 200}{3},\\
					x&>\frac{2\cdot 1000}{3}\cdot\frac{1}{x}+\frac{2\cdot 200}{3}\frac{1}{x^2},\\
					\frac{3}{2}x^3&>1000x+200,\\
					3x^3-\frac{3}{2}x^3&>1000x+200,\\
					3x^3-1000x-200&>\frac{3}{2}x^3,\\
					|3x^3-1000x-200|&>\frac{3}{2}|x^3|.
				\end{align*}
				Let $A=3/2$ and let $a=800$. Then \[
				A|x^3|\leq|3x^3-1000x-200|\quad\text{for all $x>a$}.
				\] $\therefore 3x^3-1000x-200$ is $\Omega(x^3)$.
				\item Suppose that $r<3$. Since $x^r<x^3$, we have $A|x^r|<A|x^3|$ for $x>1$. Thus \[
				A|x^r|\leq|3x^3-1000x-200|\quad\text{for all $x>a=800>1$}
				\] Hence, $3x^3-1000x-200$ is $\Omega(x^r)$ for all integer $r<3$.
			\end{enumerate}
		\end{proof}
	\end{example}
	\vspace{20pt}
	\newpage
	\begin{tcolorbox}[colframe=thmcolor,title={\color{white}\bf On Polynomial Orders}]
		\begin{theorem}
			Let $a_i\in\R$ for $i=0,\dots,n$ with $a_n\neq 0$.
			\begin{enumerate}[(1)]
				\item $\sum_{i=0}^{n}a_ix^i$ is $O(x^s)$ for all integers $s\geq n$.
				\item $\sum_{i=0}^{n}a_ix^i$ is $\Omega(x^r)$ for all integers $r\leq n$.
				\item $\sum_{i=0}^{n}a_ix^i$ is $\Theta(x^n)$.
			\end{enumerate}
		\end{theorem}
	\end{tcolorbox}
	\begin{proof}
		Let $A=\sum_{i=0}^n|a_i|$ and $a=1$. Then$\abs{\sum_{i=0}^na_ix^i}\leq A|x^n|$ for all $x>1$.
	\end{proof}
	\vspace{10pt}
	\begin{example}
		Show that $x^2$ is not $O(x)$, and deduce that $x^2$ is not $\Theta(x)$.
		\begin{proof}[\sol]
			Suppose that $x^2$ is $O(x)$. Then \begin{equation}\tag{*}
				(\exists B>0)(\exists b\geq 0)(\forall x>b)\quad|x^2|\leq B|x|.
			\end{equation} Since \[
			x\cdot x> B\cdot x\implies |x^2|> B|x|\implies\exists x>b:|x^2|> B|x|.
			\] This contradicts (*). Hence, $x^2$ is not $O(x)$.
		\end{proof}
	\end{example}
	\vspace{10pt}
	\begin{tcolorbox}[colframe=thmcolor,title={\color{white}\bf Limitation on Orders of Polynomial Functions}]
		\begin{theorem}
			Let $n\in\Z^+$, and let $a_i\in\R$ for $i=0,\dots,n$ with $a_n\neq 0$. If $m<n$, then \begin{enumerate}[(1)]
				\item $\sum_{i=0}^na_ix^i$ is not $O(x^m)$ and
				\item $\sum_{i=0}^na_ix^i$ is not $\Omega(x^m)$.
			\end{enumerate}
		\end{theorem}
	\end{tcolorbox}

	\newpage
	\section{Application: Efficiency of Algorithm I}
	\begin{note}
		Time efficiency of algorithm counting the number of elementary operations. 
	\end{note}
	\vspace{20pt}
	\begin{tcolorbox}[colframe=defcolor,title={\color{white}\bf }]
		\begin{definition}
			Let $A$ be an algorithm.
			\begin{itemize}
				\item $b(n)=$ Minimum number of elementary operation.
				\begin{itemize}
					\item If $b(n)=\Theta(g(n))$, we say in the bast case $A$ is $\Theta(g(n))$.
					\item $A$ has a best case order of $g(n)$.
				\end{itemize}
				\item $w(n)=$ Maximum number of elementary operation.
				\begin{itemize}
					\item If $w(n)=\Theta(g(n))$, we say in the worst case $A$ is $\Theta(g(n))$.
					\item $A$ has a worst case order of $g(n)$.
				\end{itemize}
			\end{itemize}
		\end{definition}
	\end{tcolorbox}
	\vspace{10pt}
	\begin{example}
		Assume $n$ is a positive integer and consider the following algorithm segment:
		\begin{figure}[h!]
			\centering
			\begin{tabular}{l}
				$p:=0$, $x:=2$\\
				\textbf{for $i:=2$ to $n$}\\
				\tab$p:=(p+i)\cdot x$\\
				\textbf{next $i$}
			\end{tabular}
		\end{figure}
		\begin{enumerate}[a.]
			\item Compute the actual number of additions and multiplications that must be performed when this algorithm segment is executed.
			\item Use the theorem on polynomial orders to find an order for this algorithm segment.
		\end{enumerate}
		\begin{proof}[\sol]
			\ \begin{enumerate}[a.]
				\item There are two operations ($+,\times$) for each iterations of the loop. The number of iterations of the \textbf{for-next} loop equals \[
				\text{the top index}\ -\ \text{the bottom index}\ + 1=n-2+1=n-1.
				\] Hence there are $2(n-1)=2n-2$ multiplications and additions.
				\item By the theorem on polynomial orders, \[
				2n-2\quad\text{is}\quad\Theta(n),
				\] and so this algorithm segment is  $\Theta(n)$.
			\end{enumerate}
		\end{proof}
	\end{example}
	
	\newpage
	\begin{example}
		Assume $n$ is a positive integer and consider the following algorithm segment:
		\begin{figure}[h!]
			\centering
			\begin{tabular}{l}
				$s:=0$\\
				\textbf{for $i:=1$ to $n$}\\
				\tab\textbf{for $j:=1$ to $i$}\\
				\tab\tab$s:=s+j\cdot(i-j+1)$\\
				\tab\textbf{next $j$}\\
				\textbf{next $i$}
			\end{tabular}
		\end{figure}
		\begin{enumerate}[a.]
			\item Compute the actual number of additions and multiplications that must be performed when this algorithm segment is executed.
			\item Use the theorem on polynomial orders to find an order for this algorithm segment.
		\end{enumerate}
		\begin{proof}[\sol]
			\ \begin{enumerate}[a.]
				\item There are four operations ($+,\cdot,-,+$) for each iterations of the loop. Note that \begin{figure}[h!]
					\centering
					\begin{tabular}{c||c|cc|ccc|c|cccc}
						\toprule[1.2pt]
						$i$ & 1 & 2 &   & 3 &   &   &$\cdots$ & $n$ &&&\\
						\hline
						\hline
						$j$ & 1 & 1 & 2 & 1 & 2 & 3 &$\cdots$ & 1 & 2 &$\cdots$&$n$\\
						\bottomrule[1.2pt]
					\end{tabular}
				\end{figure} Hence the total number of iterations of the inner loop is \[
				1+2+\cdots+n=\frac{n(n+1)}{2},
				\] and so the number of operations is $
				4\cdot\frac{n(n+1)}{2}=2n(n+1).$
				\item By the theorem on polynomial orders, $
				2n(n+1)$ is $\Theta(n^2),
				$ and so this algorithm segment is  $\Theta(n^2)$.
			\end{enumerate}
		\end{proof}
	\end{example}
	
	\begin{tcolorbox}[colframe=defcolor,title={\color{white}\bf Asymptotic Upper Bound}]
		\begin{definition}
			\[
			O(g(n)):=\set{f(n):(\exists c,n_0>0)(\forall n\geq n_0)\ 0\leq f(n)\leq cg(n)}.
			\]
		\end{definition}
	\end{tcolorbox}
	\begin{tcolorbox}[colframe=defcolor,title={\color{white}\bf Asymptotic Lower Bound}]
		\begin{definition}
			\[
			\Omega(g(n)):=\set{f(n):(\exists c,n_0>0)(\forall n\geq n_0)\ 0\leq cg(n)\leq f(n)}.
			\]
		\end{definition}
	\end{tcolorbox}
	\begin{tcolorbox}[colframe=defcolor,title={\color{white}\bf Asymptotic Tight Bound}]
		\begin{definition}
			\[
			\Theta(g(n)):=\set{f(n):(\exists c_1,c_2,n_0>0)(\forall n\geq n_0)\ 0\leq c_g(n)\leq f(n)\leq c_2g(n)}.
			\]
		\end{definition}
	\end{tcolorbox}
	
	\newpage
	\chapter{Sorting of Numbers}
	\begin{itemize}
		\item \textbf{Input:} A sequence of $n$ numbers $[a_1,a_2,\dots,a_n]$.
		\item \textbf{Output:} A sorted permutation $
		[a_1',a_2',\dots,a_n']$ \text{s.t.} $a_1'\leq a_2'\leq\cdots\leq a_n'.$
	\end{itemize}
	
	\section{The Insertion Sort Algorithm}
	
	\begin{algorithm}[H]
		\caption{Insertion-Sort $(A)$}
		\BlankLine
		\KwIn{$A=[a_1,a_2,\dots,a_n]$}
		\KwOut{$A'=[a_1',a_2',\dots,a_n']$ s.t. $a_1'\leq a_2'\leq\cdots\leq a_n'$}
		\BlankLine
		\For{$j\gets 2$ \KwTo $n$}{
			key$\gets A[j]$\;
			\Comment{Insert $A[j]$ into the sorted sequence $A[1,\dots,j-1]$}
			$i\gets j-1$\;
			\While{$i>0$ and $A[i]>\text{key}$}{
				$A[i+1]\gets A[i]$\;
				$i\gets i-1$\;
			}
			$A[i+1]\gets\text{key}$\;
		}
	\end{algorithm}
	\begin{lstlisting}[style=C]
void insertion(int* arr, int num) {
	int key;
	
	for(int i=1; i<num; i++) {
		key = *(arr+i);
		int j = i-1;
		while (j>=0 && *(arr+j) > key) {
			*(arr+j+1) = *(arr+j);
			j -= 1;
		}
		*(arr+j+1) = key;
	}
}

/************************************************
* Input: 126 062 214 103 004 098 150 055 136 077 
* 
* [i=1] 062 126 214 103 004 098 150 055 136 077 
* [i=2] 062 126 214 103 004 098 150 055 136 077 
* [i=3] 062 103 126 214 004 098 150 055 136 077 
* [i=4] 004 062 103 126 214 098 150 055 136 077 
* [i=5] 004 062 098 103 126 214 150 055 136 077 
* [i=6] 004 062 098 103 126 150 214 055 136 077 
* [i=7] 004 055 062 098 103 126 150 214 136 077 
* [i=8] 004 055 062 098 103 126 136 150 214 077 
* [i=9] 004 055 062 077 098 103 126 136 150 214 
* 
* Output: 004 055 062 077 098 103 126 136 150 214 
************************************************/
	\end{lstlisting}
	
	\subsection{Time Analysis 1}
	\begin{figure}[h!]
		\centering
		\includegraphics[scale=.45]{insert.png}
	\end{figure}
	
	\begin{itemize}
		\item The running time of Insertion-Sort on an input of $n$ values, $T(n)$, is the sum of the products of the \textit{cost} and \textit{times} columns:
		\begin{align*}
			T(n)=&c_1n+c_2(n-1)+c_4(n-1)\\
			&+c_5\sum_{j=2}^{n}t_j+c_6\sum_{j=2}^{n}(t_j-1)+c_7\sum_{j=2}^{n}(t_j-1)\\
			&+c_8(n-1).
		\end{align*}
		\item Best-case running time: \begin{align*}
			T(n)&=c_1n+c_2(n-1)+c_4(n-1)+c_5(n-1)+c_8(n-1)\\
			&=(c_1+c_2+c_4+c_5+c_8)n-(c_2+c_4+c_5+c_8).
		\end{align*}
		\item $T(n)=\Omega(n)$.
		\item Worst-case running time:
		\begin{align*}
			T(n)=&c_1n+c_2(n-1)+c_4(n-1)\\
			&+c_5\del{\frac{n(n+1)}{2}-1}+c_6\del{\frac{n(n-1)}{2}}+c_7\del{\frac{n(n-1)}{2}}\\
			&+c_8(n-1)\\
			=&\del{\frac{c_5}{2}+\frac{c_6}{2}+\frac{c_7}{2}}n^2+\del{c_1+c_2+c_4+\frac{c_5}{2}-\frac{c_6}{2}-\frac{c_7}{2}+c_8}n\\
			&-(c_2+c_4+c_5+c_8).
		\end{align*}
		\item $T(n)=O(n^2)$.
	\end{itemize}
	
	\subsection{Complexity Analysis}
	\begin{itemize}
		\item The best-case running time of insertion
		sort is $\Omega(n)$. The running time of insertion
		sort therefore falls between $\Omega(n)$ and
		$O(n^2)$. The worst-case running time of
		insertion sort is $\Omega(n^2)$, since there exists
		an input that causes the algorithm to
		take $\Omega(n^2)$ time.
		\item When we say that the running time of an
		algorithm is $\Omega(g(n))$, we mean that no
		matter what particular input size is chosen
		for each value of $n$, the running time on
		that input is at least a constant times $g(n)$,
		for sufficiently large $n$.
	\end{itemize}

	\newpage
	\section{The Bubble Sort Algorithm}
	\begin{algorithm}[H]
		\caption{Bubble-Sort $(A)$}
		\BlankLine
		\KwIn{$A=[a_1,a_2,\dots,a_n]$}
		\KwOut{$A'=[a_1',a_2',\dots,a_n']$ s.t. $a_1'\leq a_2'\leq\cdots\leq a_n'$}
		\BlankLine
		\For{$i\gets 1$ \KwTo $n$}{
			\For{$j\gets n$ \Downto $i+1$}{
				\If{$A[j]<A[j-1]$}{
					exchange $A[j]\leftrightarrow A[j-1]$\;	
				}
			}
		}
	\end{algorithm}
	
	\begin{lstlisting}[style=C]
void bubble(int* arr, int num) {
	for(int i=0; i<num; i++) {
		for(int j=num; j>i+1; j--) {
			if(*(arr+j) < *(arr+j-1)) {
				int tmp = *(arr+j);
				*(arr+j) = *(arr+j-1);
				*(arr+j-1) = tmp;
			}
		}
	}
}
/*******************************************
* Input: 92 54 99 02 47 66 66 05 20 72 
* 
* [i j = 00 09] 92 54 99 02 47 66 66 05 20 72 
* [i j = 00 08] 92 54 99 02 47 66 66 05 20 72 
* [i j = 00 07] 92 54 99 02 47 66 05 66 20 72 
* [i j = 00 06] 92 54 99 02 47 05 66 66 20 72 
* [i j = 00 05] 92 54 99 02 05 47 66 66 20 72 
* [i j = 00 04] 92 54 99 02 05 47 66 66 20 72 
* [i j = 00 03] 92 54 02 99 05 47 66 66 20 72 
* [i j = 00 02] 92 02 54 99 05 47 66 66 20 72 
* [i j = 00 01] 02 92 54 99 05 47 66 66 20 72 
* [i j = 01 09] 02 92 54 99 05 47 66 66 20 72 
* [i j = 01 08] 02 92 54 99 05 47 66 20 66 72 
* [i j = 01 07] 02 92 54 99 05 47 20 66 66 72 
* [i j = 01 06] 02 92 54 99 05 20 47 66 66 72 
* [i j = 01 05] 02 92 54 99 05 20 47 66 66 72 
* [i j = 01 04] 02 92 54 05 99 20 47 66 66 72 
* [i j = 01 03] 02 92 05 54 99 20 47 66 66 72 
* [i j = 01 02] 02 05 92 54 99 20 47 66 66 72 
* [i j = 02 09] 02 05 92 54 99 20 47 66 66 72 
* [i j = 02 08] 02 05 92 54 99 20 47 66 66 72 
* [i j = 02 07] 02 05 92 54 99 20 47 66 66 72 
* [i j = 02 06] 02 05 92 54 99 20 47 66 66 72 
* [i j = 02 05] 02 05 92 54 20 99 47 66 66 72 
* [i j = 02 04] 02 05 92 20 54 99 47 66 66 72 
* [i j = 02 03] 02 05 20 92 54 99 47 66 66 72 
* [i j = 03 09] 02 05 20 92 54 99 47 66 66 72 
* [i j = 03 08] 02 05 20 92 54 99 47 66 66 72 
* [i j = 03 07] 02 05 20 92 54 99 47 66 66 72 
* [i j = 03 06] 02 05 20 92 54 47 99 66 66 72 
* [i j = 03 05] 02 05 20 92 47 54 99 66 66 72 
* [i j = 03 04] 02 05 20 47 92 54 99 66 66 72 
* [i j = 04 09] 02 05 20 47 92 54 99 66 66 72 
* [i j = 04 08] 02 05 20 47 92 54 99 66 66 72 
* [i j = 04 07] 02 05 20 47 92 54 66 99 66 72 
* [i j = 04 06] 02 05 20 47 92 54 66 99 66 72 
* [i j = 04 05] 02 05 20 47 54 92 66 99 66 72 
* [i j = 05 09] 02 05 20 47 54 92 66 99 66 72 
* [i j = 05 08] 02 05 20 47 54 92 66 66 99 72 
* [i j = 05 07] 02 05 20 47 54 92 66 66 99 72 
* [i j = 05 06] 02 05 20 47 54 66 92 66 99 72 
* [i j = 06 09] 02 05 20 47 54 66 92 66 72 99 
* [i j = 06 08] 02 05 20 47 54 66 92 66 72 99 
* [i j = 06 07] 02 05 20 47 54 66 66 92 72 99 
* [i j = 07 09] 02 05 20 47 54 66 66 92 72 99 
* [i j = 07 08] 02 05 20 47 54 66 66 72 92 99 
* [i j = 08 09] 02 05 20 47 54 66 66 72 92 99 
* 
* Output: 02 05 20 47 54 66 66 72 92 99 
*******************************************/
	\end{lstlisting}
	
	
	
	
	
	\newpage
	\section{Sorting Networks}
	\begin{tcolorbox}[colframe=defcolor,title={\color{white}\bf Comparator}]
		\begin{definition}
			A \textbf{comparator} is a device with two inputs, \(x\) and $y$, and two outputs, \(x'\) and $y'$, that performs the following function: \begin{align*}
				x'=\min(x,y),\\
				y'=\max(x,y).
			\end{align*}
			\begin{center}
				\begin{tikzpicture}
				\foreach \y/\startlabel/\endlabel in {1/{$x$}/{$x'=\min(x,y)$}, 0/{$y$}/{$y'=\max(x,y)$}} {
					\draw (4,\y) -- (0,\y) node[anchor=east] {\startlabel};
					\draw (4,\y) -- (8,\y) node[anchor=west] {\endlabel};
				}
				
				% Comparators (1st and 2nd, 3rd and 4th) with dots
				\draw (4,0) -- (4,1);
				\fill (4,0) circle (2pt);
				\fill (4,1) circle (2pt);
			\end{tikzpicture}
			\end{center}
		\end{definition}
	\end{tcolorbox}
	\begin{remark}
		Works in \(O(1)\) time.
	\end{remark}
	\vspace{4pt}
	\begin{example}
		\ \begin{center}
			\begin{tikzpicture}
				\foreach \y/\l [count=\n from 1] in {3/a, 2/a, 1/a, 0/a} {
					\fill (0,\y) node[anchor=east] {$\l_\n$};
					\fill (10,\y) node[anchor=west] {$b_\n$};
				}
				\foreach \y/\labels in {
					3/{9,5,2,,2},
					2/{5,9,,6,5},
					1/{2,2,5,,6},
					0/{6,6,,9,9},
				}{
					\foreach \label [count=\xi from 0] in \labels {
						\draw (2*\xi,\y) -- (2*\xi+2,\y) node[midway,above] {\label};
					}
				}
				%1
				\draw (2,3) -- (2,2);
				\fill (2,3) circle (2pt);
				\fill (2,2) circle (2pt);
				\draw (2,1) -- (2,0);
				\fill (2,1) circle (2pt);
				\fill (2,0) circle (2pt);
				%2
				\draw (4,3) -- (4,1);
				\fill (4,3) circle (2pt);
				\fill (4,1) circle (2pt);
				%3
				\draw (6,2) -- (6,0);
				\fill (6,2) circle (2pt);
				\fill (6,0) circle (2pt);
				%4
				\draw (8,2) -- (8,1);
				\fill (8,2) circle (2pt);
				\fill (8,1) circle (2pt);
				%depth
				\fill (0,-1) node[] {depth};
				\fill (2,-1) node[] {$1$};
				\fill (4,-1) node[] {$2$};
				\fill (6,-1) node[] {$2$};
				\fill (8,-1) node[] {$3$};
			\end{tikzpicture}
		\end{center}
		\begin{itemize}
			\item Wires go straight, left to right.
			\item Each comparator has inputs/outputs on some pair of wires.
		\end{itemize}
	\end{example}
	\vspace{8pt}
	\begin{tcolorbox}[colframe=defcolor,title={\color{white}\bf Depth}]
		\begin{definition}
			We define the \textbf{depth} of a wire as follows:
			\begin{enumerate}[(1)]
				\item An input wire of a comparison network has depth \(0\).
				\item If a comparator has two input wires with depths \(d_x\) and \(d_y\), then its output wires have depth \(\max(d_x,d_y)+1\).
			\end{enumerate}
		\end{definition}
	\end{tcolorbox}

	\newpage
	\begin{remark}
		\ \begin{enumerate}[(1)]
			\item \textbf{Depth of a Comparator} := depth of its output wire.
			\item \textbf{Depth of a Network} := maximum depth of an output of the network.
		\end{enumerate}
	\end{remark}
	\vspace{8pt}
	\begin{example}[Bouble-Sort]
		Find the maximum of 5 values:\begin{center}
			\begin{tikzpicture}
				\foreach \i in {0,...,4} {
					\draw (0,\i) -- (10,\i);
				}
				%1
				\draw (2,4) -- (2,3);
				\fill (2,4) circle (2pt);
				\fill (2,3) circle (2pt);
				%2
				\draw (4,3) -- (4,2);
				\fill (4,3) circle (2pt);
				\fill (4,2) circle (2pt);
				%3
				\draw (6,2) -- (6,1);
				\fill (6,2) circle (2pt);
				\fill (6,1) circle (2pt);
				%4
				\draw (8,1) -- (8,0);
				\fill (8,1) circle (2pt);
				\fill (8,0) circle (2pt);
				%depth
			\end{tikzpicture}
		\end{center}
		We extend our idea:
		\begin{center}
			\begin{tikzpicture}
				\foreach \i in {0,...,4} {
					\draw (0,\i) -- (10,\i);
				}
				\foreach \i in {2,4,6,8} {
					\draw (\i,4) -- (\i,3);
					\fill (\i,4) circle (2pt);
					\fill (\i,3) circle (2pt);
				}
				\foreach \i in {3,5,7} {
					\draw (\i,3) -- (\i,2);
					\fill (\i,3) circle (2pt);
					\fill (\i,2) circle (2pt);
				}
				\foreach \i in {4,6} {
					\draw (\i,2) -- (\i,1);
					\fill (\i,2) circle (2pt);
					\fill (\i,1) circle (2pt);
				}
				\draw (5,1) -- (5,0);
				\fill (5,1) circle (2pt);
				\fill (5,0) circle (2pt);
				%\draw[thick, red] (5,5) -- (10,1);
				%\draw[thick, red] (3,5) -- (9,0);
				\draw[thick, red] (1,5) -- (8,-1);
				\fill[blue] (2,3) circle (3pt);
				\fill[blue] (3,3) circle (3pt);
			\end{tikzpicture}
		\end{center}
		\textbf{Depth:} \[
		\begin{cases}
			D(n)=D(n-1)+\textcolor{blue}{2}\\
			D(2)=1
		\end{cases}.
		\] Then \begin{align*}
			D(2)&=1\\
			D(3)&=D(2)+2=3\\
			D(4)&=D(2)+2+2=5\\
			D(5)&=D(2)+2+2+2=7\\
			\vdots\\
			D(k)&=D(2)+2\cdot(k-2),
		\end{align*} and so \[
		D(n)=D(2)+2\cdot(n-2)=1+2n-4=2n-3=\Theta(n).
		\]
	\end{example}

	\subsection{The Zero-One Principle}
	\begin{quote}
		``How can we test if a comparison network sorts?''
	\end{quote}
	\vspace{8pt}
	\begin{tcolorbox}[colframe=lemcolor,title={\color{white}\bf }]
		\begin{lemma}
			If a comparison network transforms \[
			a=\langle a_1,\dots,a_n\rangle\quad\text{into}\quad b=\langle b_1,\dots,b_n\rangle,
			\] then for any montonically increasing function \(f\), it transforms \[
			f(a)=\langle f(a_1),\dots,f(a_n)\rangle\quad\text{into}\quad f(b)=\langle f(b_1),\dots,f(b_n)\rangle.
			\]
		\end{lemma}
	\end{tcolorbox}
	\begin{proof}
		Since \(f\) is montotonically increasing function, we know \begin{center}
			\begin{tikzpicture}
				\foreach \y/\startlabel/\endlabel in {1/{$f(x)$}/{$\min(f(x),f(y))=f(\min(x,y))$}, 0/{$f(y)$}/{$\max(f(x),f(y))=f(\max(x,y))$}} {
					\draw (4,\y) -- (0,\y) node[anchor=east] {\startlabel};
					\draw (4,\y) -- (8,\y) node[anchor=west] {\endlabel};
				}
				
				% Comparators (1st and 2nd, 3rd and 4th) with dots
				\draw (4,0) -- (4,1);
				\fill (4,0) circle (2pt);
				\fill (4,1) circle (2pt);
			\end{tikzpicture}
		\end{center} We use mathematical induction on the depth of each wire in a general comparison network: \begin{enumerate}[(i)]
		\item (Basis Step) Clearly
		\item (Induction Step)
	\end{enumerate}
	\end{proof}
	\vspace{8pt}
	\begin{example}
		Consider \begin{center}
			\begin{tikzpicture}
				\foreach \y/\startlabel/\endlabel in {0/3/6, 1/4/5, 2/2/4, 3/5/3, 4/6/2} {
					\draw (4,\y) -- (0,\y) node[anchor=east] {\startlabel};
					\draw (4,\y) -- (8,\y) node[anchor=west] {\endlabel};
				}
			\end{tikzpicture}
		\end{center}
		Let \[
		f(x):=\begin{cases}
			0 &:x\leq 3\\
			1 &:x> 1
		\end{cases}
		\] Then \begin{center}
			\begin{tikzpicture}
				\foreach \y/\startlabel/\endlabel in {0/{$f(3)=0$}/{$1=f(6)$}, 1/{$f(4)=1$}/{$1=f(5)$}, 2/{$f(2)=0$}/{$1=f(4)$}, 3/{$f(5)=1$}/{$0=f(3)$}, 4/{$f(6)=1$}/{$0=f(2)$}} {
					\draw (4,\y) -- (0,\y) node[anchor=east] {\startlabel};
					\draw (4,\y) -- (8,\y) node[anchor=west] {\endlabel};
				}
			\end{tikzpicture}
		\end{center}
	\end{example}
	\vspace{8pt}
	\begin{tcolorbox}[colframe=thmcolor,title={\color{white}\bf Zero-One Principle}]
		\begin{theorem}
			If a comparison network with \(n\) inputs sorts all \(2^n\) possible sequences of \(0\)'s and \(1's\), then it sorts all sequences of arbitrary numbers correctly.
		\end{theorem}
	\end{tcolorbox}
	\begin{proof}
		Suppose that \[
		\exists\text{sequence}\ \langle a_1,a_2,\dots,a_n\rangle\ \text{s.t.}\ a_i<a_j\ \text{but}\ \langle \cdots,a_j,\cdots,a_i,\cdots\rangle.
		\] We define a montonically increasing function \(f\) as \[
		f(x):=\begin{cases}
			0&:x\leq a_i,\\
			1&:x> a_i.
		\end{cases}
		\] By Lemma,
	\end{proof}

	\newpage
	\subsection{A Bitonic Sorting Network}
	\begin{tcolorbox}[colframe=defcolor,title={\color{white}\bf Bitonic}]
		\begin{definition}
			A sequence is \textbf{bitonic} if it monotonically increases, then monotonically decreases, or it can be circularly shifted to become so.
		\end{definition}
	\end{tcolorbox}
	\vspace{20pt}
	
	\begin{tcolorbox}[colframe=defcolor,title={\color{white}\bf Half-cleaner}]
		\begin{definition}A bitonic sorter is composed of severs stages, each of which is called a \textbf{half -cleaner}. Each half-cleaner is a comparison network of depth $1$ in which input line $i$ is compared with line $i+\frac{n}{2}$ for $i=1,2\dots,\frac{n}{2}$. (We assume that $n$ is even.)
					
		\end{definition}
	\end{tcolorbox}
	\begin{example}
		\ \begin{figure}[h!]
			\centering
			\begin{minipage}{.495\textwidth}
				\begin{tikzpicture}[scale=.6]
					\foreach \y/\startlabel/\endlabel in {
						0/0/1, 1/0/1, 2/0/0, 3/1/1,
						4/1/0, 5/1/0, 6/0/0, 7/0/0
					} {
						\draw (2.5,\y) -- (0,\y) node[anchor=east] {\startlabel};
						\draw (2.5,\y) -- (5,\y) node[anchor=west] {\endlabel};
					}
					\foreach \x/\y in {1/3, 2/2, 3/1, 4/0} {
						\fill (\x,\y) circle (3pt);
						\draw (\x,\y) -- (\x,\y+4);
						\fill (\x,\y+4) circle (3pt);
					}
					\draw (-1, 7) -- (-1,0) node[midway, left] {bitonic};
					\draw (-1, 0) -- (-.75,0);
					\draw (-1, 7) -- (-.75,7);
					\draw (5.75, 0) -- (6,0);
					\draw (5.75, 7) -- (6,7);
					\draw (5.75, 4) -- (6,4);
					\draw (5.75, 3) -- (6,3);
					\draw (6, 0) -- (6,3) node[midway, right] {bitonic};
					\draw (6, 4) -- (6,5.5) node[right] {clean};
					\draw (6, 5.5) -- (6,7) node[midway, right] {bitonic,};
				\end{tikzpicture}
				\end{minipage}
			\begin{minipage}{.495\textwidth}
				\begin{tikzpicture}[scale=.6]
					\foreach \y/\startlabel/\endlabel in {
						0/0/1, 1/1/1, 2/1/1, 3/1/1,
						4/1/0, 5/1/1, 6/0/0, 7/0/0
					} {
						\draw (2.5,\y) -- (0,\y) node[anchor=east] {\startlabel};
						\draw (2.5,\y) -- (5,\y) node[anchor=west] {\endlabel};
					}
					\foreach \x/\y in {1/3, 2/2, 3/1, 4/0} {
						\fill (\x,\y) circle (3pt);
						\draw (\x,\y) -- (\x,\y+4);
						\fill (\x,\y+4) circle (3pt);
					}
					\draw (-1, 7) -- (-1,0) node[midway, left] {bitonic};
					\draw (-1, 0) -- (-.75,0);
					\draw (-1, 7) -- (-.75,7);
					\draw (5.75, 0) -- (6,0);
					\draw (5.75, 7) -- (6,7);
					\draw (5.75, 4) -- (6,4);
					\draw (5.75, 3) -- (6,3);
					\draw (6, 1.5) -- (6,3) node[midway, right] {bitonic,};
					\draw (6, 0) -- (6,1.5) node[right] {clean};
					\draw (6, 4) -- (6,7) node[midway, right] {bitonic};
				\end{tikzpicture}
			\end{minipage}
			\caption{The comparison network $\mathsf{Half}$-$\mathsf{Cleaner}[8]$.}
		\end{figure}
	\end{example}
	\vspace{10pt}
	\begin{tcolorbox}[colframe=lemcolor,title={\color{white}\bf }]
		\begin{lemma}
			Let the input to a half-cleaner is a bitonic sequence of 0's and 1's. Then the output satisfies the following properties:
			\begin{enumerate}[(1)]
				\item both the top half and the bottom half are bitonic;
				\item every element in the top half is at least as small as every element of the bottom half, and
				\item at least one half is \textbf{clean}(all 0's or all 1's).
			\end{enumerate}
		\end{lemma}
	\end{tcolorbox}
	\begin{proof}
		content...
	\end{proof}
	
	\begin{example}
		\ \begin{figure}[h!]
			\begin{tikzpicture}[scale=1]
				\fill[yellow] (0.75,-.25) rectangle (4.25,7.25);
				\fill[yellow] (5.75,3.75) rectangle (7.25,7.25);
				\fill[yellow] (5.75,-.25) rectangle (7.25,3.25);
				\foreach \i in {0,2,4,6}
					\fill[yellow] (8.75, -.25+\i) rectangle (9.25, 1.25+\i);
				\foreach \y/\startlabel/\endlabel in {
					0/0, 1/0, 2/0, 3/1,
					4/1, 5/1, 6/0, 7/0
				}{
					\draw (4,\y) -- (0,\y) node[anchor=east] {\startlabel};
					%\draw (2.5,\y) -- (5,\y) node[anchor=west] {\endlabel};
				}
				\foreach \y/\label in {
					0/1, 1/1, 2/0, 3/1,
					4/0, 5/0, 6/0, 7/0
				}{
					\draw (4,\y) -- (6,\y) node[midway,above] {\label};
				}
				\foreach \x/\y in {1/3, 2/2, 3/1, 4/0} {
					\fill (\x,\y) circle (3pt);
					\draw (\x,\y) -- (\x,\y+4);
					\fill (\x,\y+4) circle (3pt);
				}
				\foreach \i in {0,...,7}
					\draw (6,\i) -- (8,\i);
				\foreach \x/\y in {6/1, 7/0} {
					\fill (\x,\y) circle (3pt);
					\draw (\x,\y) -- (\x,\y+2);
					\fill (\x,\y+2) circle (3pt);
					\fill (\x,\y+4) circle (3pt);
					\draw (\x,\y+4) -- (\x,\y+6);
					\fill (\x,\y+6) circle (3pt);
				}\foreach \y/\label in {
					0/1, 1/1, 2/0, 3/1,
					4/0, 5/0, 6/0, 7/0
				}{
					\draw (7,\y) -- (9,\y) node[midway,above] {\label};
				}
				\foreach \i in {0,2,4,6} {
					\fill (9,\i) circle (3pt);
					\fill (9,\i+1) circle (3pt);
					\draw (9,\i) -- (9,\i+1);
				}
				\foreach \i/\j in {
				0/1,1/1,2/1,3/0,
				4/0,5/0,6/0,7/0
				}{
					\draw (9,\i) -- (10,\i) node[anchor=west] {\j};
				}
				\draw (-1, 7) -- (-1,0) node[midway, left] {bitonic};
				\draw (-1, 0) -- (-.75,0);
				\draw (-1, 7) -- (-.75,7);
				\draw (10.75, 0) -- (10.5,0);
				\draw (10.75, 7) -- (10.5,7);
				\draw (10.75, 0) -- (10.75,7) node[midway, right] {sorted};
			\end{tikzpicture}
		\end{figure}
	\end{example}

	\newpage
	\subsection{Merging Network}
	
	\begin{example}
		\ \begin{figure}[h!]
			\begin{tikzpicture}[scale=1]
				\fill[yellow] (0.75,-.25) rectangle (4.25,7.25);
				\fill[yellow] (5.75,3.75) rectangle (7.25,7.25);
				\fill[yellow] (5.75,-.25) rectangle (7.25,3.25);
				\foreach \i in {0,2,4,6}
				\fill[yellow] (8.75, -.25+\i) rectangle (9.25, 1.25+\i);
				\foreach \y/\startlabel/\endlabel in {
					0/1, 1/1, 2/1, 3/0,
					4/1, 5/1, 6/0, 7/0
				}{
					\draw (4,\y) -- (0,\y) node[anchor=east] {\startlabel};
					%\draw (2.5,\y) -- (5,\y) node[anchor=west] {\endlabel};
				}
				\foreach \y/\label in {
					0/1, 1/1, 2/1, 3/1,
					4/0, 5/0, 6/1, 7/0
				}{
					\draw (4,\y) -- (6,\y) node[midway,above] {\label};
				}
				\foreach \x/\y/\z in {1/3/0, 2/2/1, 3/1/2, 4/0/3} {
					\fill (\x,\z) circle (3pt);
					\fill (\x,7-\z) circle (3pt);
					\draw (\x,\z) -- (\x,7-\z);
				}
				\foreach \i in {0,...,7}
				\draw (6,\i) -- (8,\i);
				\foreach \x/\y in {6/1, 7/0} {
					\fill (\x,\y) circle (3pt);
					\draw (\x,\y) -- (\x,\y+2);
					\fill (\x,\y+2) circle (3pt);
					\fill (\x,\y+4) circle (3pt);
					\draw (\x,\y+4) -- (\x,\y+6);
					\fill (\x,\y+6) circle (3pt);
				}\foreach \y/\label in {
					0/1, 1/1, 2/1, 3/1,
					4/1, 5/0, 6/0, 7/0
				}{
					\draw (7,\y) -- (9,\y) node[midway,above] {\label};
				}
				\foreach \i in {0,2,4,6} {
					\fill (9,\i) circle (3pt);
					\fill (9,\i+1) circle (3pt);
					\draw (9,\i) -- (9,\i+1);
				}
				\foreach \i/\j in {
					0/1,1/1,2/1,3/1,
					4/1,5/0,6/0,7/0
				}{
					\draw (9,\i) -- (10,\i) node[anchor=west] {\j};
				}
				\draw (-1, 7) -- (-1,4.1) node[midway, left] {sorted};
				\draw (-1, 3.9) -- (-1,0) node[midway, left] {sorted};
				\draw (-1, 0) -- (-.75,0);
				\draw (-1, 7) -- (-.75,7);
				\draw (-1, 3.9) -- (-.75,3.9);
				\draw (-1, 4.1) -- (-.75,4.1);
				\draw (10.75, 0) -- (10.5,0);
				\draw (10.75, 7) -- (10.5,7);
				\draw (10.75, 0) -- (10.75,7) node[midway, right] {sorted};
			\end{tikzpicture}
		\end{figure}
	\end{example}
	
	
	\newpage
	\chapter{Further Efficient Algorithms}
	\begin{tcolorbox}[colframe=defcolor,title={\color{white}\bf }]
		\begin{definition}
			
		\end{definition}
	\end{tcolorbox}
	\begin{tcolorbox}[colframe=thmcolor,title={\color{white}\bf }]
		\begin{theorem}
			
		\end{theorem}
	\end{tcolorbox}

	
	\begin{tikzpicture}
		
		% Input lines with dots and labels
		\foreach \y/\label in {0/A,1/B,2/C,3/D} {
			\draw (0,\y) -- (8,\y);
			\fill (0,\y) circle (2pt) node[anchor=east] {\label};
			\fill (8,\y) circle (2pt) node[anchor=west] {\label'};
		}
		%1
		\draw (1.5,0) -- (1.5,2);
		\fill (1.5,0) circle (2pt);
		\fill (1.5,2) circle (2pt);
		
		\draw (2.5,1) -- (2.5,3);
		\fill (2.5,1) circle (2pt);
		\fill (2.5,3) circle (2pt);
		
		% Comparators (1st and 2nd, 3rd and 4th) with dots
		\draw (4.5,0) -- (4.5,1);
		\fill (4.5,0) circle (2pt);
		\fill (4.5,1) circle (2pt);
		
		\draw (5.5,2) -- (5.5,3);
		\fill (5.5,2) circle (2pt);
		\fill (5.5,3) circle (2pt);
		
		% Comparators (2nd and 3rd) with dots
		\draw (7,1) -- (7,2);
		\fill (7,1) circle (2pt);
		\fill (7,2) circle (2pt);
		
	\end{tikzpicture}
	\newpage
	\chapter{Turing Machine}
	\section{Deterministic Turing Machine, The class P}
	\section{Non-deterministic Turing Machine, The class NP}
	
	\newpage
	\chapter{NP Problem}
	\section{Decidability (The Halting Problem)}
	\section{NP-completeness Theory}
	% End document
\end{document}
