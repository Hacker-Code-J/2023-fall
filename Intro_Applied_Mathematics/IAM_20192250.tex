\documentclass[12pt,openany]{book}

\usepackage{amsmath,amsthm,amsfonts,amscd} % Packages for mathematics

% Colors
\usepackage[dvipsnames]{xcolor}
\definecolor{titleblue}{RGB}{0,53,128}
\definecolor{chaptergray}{RGB}{140,140,140}
\definecolor{sectiongray}{RGB}{180,180,180}

\definecolor{thmcolor}{RGB}{231, 76, 60}
\definecolor{defcolor}{RGB}{52, 152, 219}
\definecolor{lemcolor}{RGB}{155, 89, 182}
\definecolor{corcolor}{RGB}{46, 204, 113}
\definecolor{procolor}{RGB}{241, 196, 15}

% Fonts
\usepackage[T1]{fontenc}
\usepackage[utf8]{inputenc}
\usepackage{newpxtext,newpxmath}
\usepackage{sectsty}
\allsectionsfont{\sffamily\color{titleblue}\mdseries}

% Page layout
\usepackage{geometry}
\geometry{a4paper,left=1.325in,right=1in,top=1in,bottom=1in,heightrounded}
\usepackage{fancyhdr}
\fancyhf{}
\fancyhead[LE,RO]{\thepage}
\fancyhead[LO]{\nouppercase{\rightmark}}
\fancyhead[RE]{\nouppercase{\leftmark}}
\renewcommand{\headrulewidth}{0.5pt}
\renewcommand{\footrulewidth}{0pt}

% Chapter formatting
\usepackage{titlesec}
\titleformat{\chapter}[display]
{\normalfont\sffamily\Huge\bfseries\color{titleblue}}{\chaptertitlename\ \thechapter}{20pt}{\Huge}
\titleformat{\section}
{\normalfont\sffamily\Large\bfseries\color{titleblue!100!gray}}{\thesection}{1em}{}
\titleformat{\subsection}
{\normalfont\sffamily\large\bfseries\color{titleblue!75!gray}}{\thesubsection}{1em}{}

% Table of contents formatting
\usepackage{tocloft}
\renewcommand{\cftchapfont}{\sffamily\color{titleblue}\bfseries}
\renewcommand{\cftsecfont}{\sffamily\color{chaptergray}}
\renewcommand{\cftsubsecfont}{\sffamily\color{sectiongray}}
\renewcommand{\cftchapleader}{\cftdotfill{\cftdotsep}}

% Hyperlinks
\usepackage{hyperref}
\hypersetup{
	colorlinks=true,
	linkcolor=titleblue,
	filecolor=black,      
	urlcolor=titleblue,
}

%Listing
\usepackage{listings} %Code
\renewcommand{\lstlistingname}{Code}%

\definecolor{sagegreen}{rgb}{0.0,0.6,0.4}
\definecolor{sagepurple}{rgb}{0.6,0.0,0.4}
\definecolor{sageblue}{rgb}{0.0,0.4,0.6}
\definecolor{sageorange}{rgb}{1.0,0.4,0.0}
\definecolor{sagegray}{rgb}{0.4,0.4,0.4}

\lstdefinestyle{sage}{
	language=Python,
	backgroundcolor=\color{white},
	basicstyle=\small\ttfamily\color{black}, 
	basicstyle=\footnotesize\ttfamily\color{black},
	keywordstyle=\color{blue!60!black},
	commentstyle=\color{green!60!black},
	stringstyle=\color{purple!60!black},
	showstringspaces=false,
	breaklines=true,
	tabsize=4,
	morekeywords={True, False, None},
	frame=leftline, % Remove the border
	framesep=3pt,
	frameround=tttt,
	framexleftmargin=3pt,
	numbers=left,
	numberstyle=\small\color{gray},
	xleftmargin=15pt, % Increase the left margin
	xrightmargin=5pt,
	captionpos=b,
	belowskip=0pt,
	aboveskip=4pt
}

%Ceiling and Floor Function
\usepackage{mathtools}
\DeclarePairedDelimiter{\ceil}{\lceil}{\rceil}
\DeclarePairedDelimiter{\floor}{\lfloor}{\rfloor}

%Algorithm
\usepackage[ruled,linesnumbered]{algorithm2e}
\usepackage{setspace}
\usepackage{algpseudocode}
\SetKwComment{Comment}{/* }{ */}
\SetKw{Break}{break}
\SetKw{Downto}{downto}
\SetKwProg{Fn}{Function}{:}{end}
\SetKwFunction{KeyGen}{KeyGen}


%---------------------------My Preamble
\usepackage{marvosym} %Lightning
\usepackage{booktabs}
\usepackage{multicol}
\setlength{\columnsep}{2cm}
\setlength{\columnseprule}{1.25pt}
\usepackage{enumerate}
\usepackage{soul}
\newcommand{\mathcolorbox}[2]{\colorbox{#1}{$\displaystyle #2$}}
\usepackage{graphicx}
\usepackage{tikz}
\usepackage{tikz-cd}
\usetikzlibrary{calc}
\usetikzlibrary{arrows, arrows.meta, positioning, shapes.multipart}
\usepackage{pgfplots}

%Tcolorbox
\usepackage[most]{tcolorbox}
\tcbset{colback=white, arc=5pt}
%\tcbset{enhanced, colback=white,colframe=black,fonttitle=\bfseries,arc=4mm,boxrule=1pt,shadow={2mm}{-1mm}{0mm}{black!50}}
%White box with black text and shadow
%\begin{tcolorbox}[colback=white,colframe=black,fonttitle=\bfseries,title=Black Shadow Box,arc=4mm,boxrule=1pt,shadow={2mm}{-1mm}{0mm}{black!50}]
%	This is a white box with black text and a subtle shadow. The shadow adds some depth and dimension to the box without overpowering the design.
%\end{tcolorbox}

%Theorem
\newtheorem{axiom}{Axiom}[chapter]
\newtheorem{theorem}{Theorem}[chapter]
\newtheorem{proposition}[theorem]{Proposition}
\newtheorem{corollary}{Corollary}[theorem]
\newtheorem{lemma}[theorem]{Lemma}

\theoremstyle{definition}
\newtheorem{definition}{Definition}[chapter]
\newtheorem{remark}{Remark}[chapter]
\newtheorem{exercise}{Exercise}[chapter]
\newtheorem{example}{Example}[chapter]
\newtheorem*{note}{Note}

%New Command
\newcommand{\set}[1]{\left\{#1\right\}}
\newcommand{\N}{\mathbb{N}}
\newcommand{\Z}{\mathbb{Z}}
\newcommand{\Q}{\mathbb{Q}}
\newcommand{\R}{\mathbb{R}}
\newcommand{\C}{\mathbb{C}}
\newcommand{\F}{\mathbb{F}}

\newcommand{\ie}{\textnormal{i.e.}}
\newcommand{\eg}{\textnormal{e.g.}}

\newcommand{\of}[1]{\left( #1 \right)} 
\newcommand{\abs}[1]{\left\lvert #1 \right\rvert}

\newcommand{\nbhd}{\mathcal{N}}
\newcommand{\Id}{\operatorname{\textnormal{id}}}

\newcommand{\sol}{\textcolor{magenta}{\bf Sol}}

\newcommand{\Caratheodroy}{Carath\'{e}odory}

% Begin document
\begin{document}
	
	% Title page
	\begin{titlepage}
		\begin{center}
			{\Huge\textsf{\textbf{Introduction to Applied Mathematics}}\par}
			\vspace{0.5in}
			{\Large Ji Yong-Hyeon\par}
			\vspace{1in}
			\includegraphics[scale=1.75]{iam2.jpg}\par
			\vspace{1in}
			{\bf Department of Information Security, Cryptology, and Mathematics\par}
			{College of Science and Technology\par}
			{Kookmin University\par}
			%\includegraphics[width=1.5in]{school_logo.jpg}\par
			\vspace{.25in}
			{\large \today\par}
		\end{center}
	\end{titlepage}
	
	% Table of contents
	\tableofcontents
	
	% Chapters
	\mainmatter
	
	\chapter{Differentiation}
	\section{Derivative and \Caratheodroy's Theorem}
	\begin{tcolorbox}[colback=white,colframe=defcolor,arc=5pt,title={\color{white}\bf Derivative}]
		\begin{definition}
			Let $f:I\to\R$ and $a\in I$. We say that $L\in\R$ is the \textbf{derivative of} $f$ at $a$ if \[
			\forall\epsilon>0:\exists\delta>0:x\in\nbhd_\delta^*(a)\cap I\implies\abs{\frac{f(x)-f(a)}{x-a}-L}<\epsilon.
			\]
		\end{definition}
	\end{tcolorbox}
	\begin{remark}
		We say that $f$ is \textbf{differentiable} at $a$, and we write $L=f'(a)$. In other words, $\displaystyle f'(a)=\lim\limits_{x\to a} \frac{f(x)-f(a)}{x-a}$.
	\end{remark}
	\vspace{8pt}
	\begin{tcolorbox}[colback=white,colframe=procolor,arc=5pt,title={\color{white}\bf }]
		\begin{proposition}
			If $f:I\to\R$ has a derivative at $a\in I$ then $f$ is continuous at $a$. That is, \[
			\exists f'(a)\implies f(a)=\lim\limits_{x\to a}f(x).
			\]
		\end{proposition}
	\end{tcolorbox}
	\begin{proof}
		Let $\exists f'(a)$. Then \begin{align*}
			\lim\limits_{x\to a} [f(x)-f(a)]&=\lim\limits_{x\to a}\left[\frac{f(x)-f(a)}{x-a}\cdot (x-a)\right]\quad\because x\in\nbhd_\delta^*(a)\Rightarrow x\neq a\\
			&=\lim\limits_{x\to a}\frac{f(x)-f(a)}{x-a}\lim\limits_{x\to a}(x-a)\\
			&=f'(a)\cdot 0=0.
		\end{align*}
	\end{proof}
	\vspace{4pt}
	\begin{remark}
		The continuity of $f:I\to\R$ at point does not assure the existence of the derivative at that point, \eg, $f(x):=\abs{x}$ for $x\in\R$.
	\end{remark}
	
	\begin{tcolorbox}[colback=white,colframe=thmcolor,arc=5pt,title={\color{white}\bf $\star$ \Caratheodroy's Theorem $\star$}]
		\begin{theorem}
			Let $f$ be defined on an interval $I$ containing the point $a$. Then \[
			\exists f'(a)\iff\exists\varphi\in\R^I\quad\textnormal{such that}\quad
			\begin{cases}
				\textnormal{$\varphi$ is continuous on $I$}&\cdots\textnormal{(1)}\\
				\\
				f(x)-f(a)=\varphi(x)(x-a)&\cdots\textnormal{(2)}
			\end{cases}
			\] In this case, we have $\varphi(a)=f'(a)$.
		\end{theorem}
	\end{tcolorbox}
	\begin{proof}
		\begin{itemize}
			\item[($\Rightarrow$)] Assume that $\exists f'(a)$. Define a function $\varphi:I\to\R$ as following \[
			\varphi(x)=\begin{cases}
				\displaystyle\frac{f(x)-f(a)}{x-a} &:x\neq a\\
				\\
				f'(a) &:x=a.
			\end{cases}
			\] Then \begin{enumerate}[(i)]
				\item $\phi$ is continuous on $I$, \ie, for all $a\in I$, $$\lim\limits_{x\to a}\varphi(x)=\lim\limits_{x\to a}\frac{f(x)-f(a)}{x-a}=f'(a)=\varphi(a).$$
				\item \[
				\begin{cases}
					f(x)-f(a)=\varphi(x)(x-a) &: x\neq a\\
					0=\varphi(x)\cdot 0 &: x=a.
				\end{cases}
				\]
			\end{enumerate}
			\item[($\Leftarrow$)] Let $x\neq a$ and $x\to a$. The continuity of $\varphi$ gives that \[
			\exists\phi(a)=\lim\limits_{x\to a}\varphi(x)=\lim\limits_{x\to a}\frac{\varphi(x)(x-a)}{(x-a)}=\lim\limits_{x\to a}\frac{f(x)-f(a)}{x-a}=f'(a).
			\] That is, $f$ is differentiable at $a$ and $f'(a)=\varphi(a)$.
		\end{itemize}
	\end{proof}
	\vspace{4pt}
	\begin{example}
		Let us consider the function $f$ defined by $f(x):=x^3$ for $x\in\R$. For any $a\in\R$, we see from the factorization \[
		f(x)-f(a)=x^3-a^3=(x^2+ax+a^2)(x-a)
		\] that $\varphi(x):=x^2+ax+a^2$ satisfies the condition of \Caratheodroy's Theorem. Therefore, we conclude that $f$ is differentiable at $a\in\R$ and that $f'(a)=\varphi(a)=3a^2$.
	\end{example}
	
	\newpage
	\begin{tcolorbox}[colback=white,colframe=thmcolor,arc=5pt,title={\color{white}\bf Chain Rule}]
		\begin{theorem}
			Let $I,J$ be intervals in $\R$, let $g:J\to\R$ and $f:I\to\R$ be functions such that $f[I]\subseteq J$, and let $a\in I$. Then \[
			\exists f'(a)\exists g'(f(a))\implies\exists(g\circ f)'(a)
			\] and $(g\circ f)'(a)=g'(f(a))f'(a)$.
		\end{theorem}
	\end{tcolorbox}
	\begin{proof}
		We must show that there exists a continuous function $\varphi(x)$ s.t. \[
		g(f(x))-g(f(a))=\varphi(x)(x-a).
		\]
		\begin{enumerate}[(1)]
			\item Since $\exists f'(a)$, by \Caratheodroy's Theorem, $\exists\sigma:I\to\R$ s.t. \begin{enumerate}[(i)]
				\item $\sigma$ is continuous at $a\in I$;
				\item $f(x)-f(a)=\sigma(x)(x-a)$;
				\item $f'(a)=\sigma(a)$.
			\end{enumerate}
			\item Since $\exists g'(f(a))$, by \Caratheodroy's Theorem, $\exists\tau:J\to\R$ s.t. \begin{enumerate}[(i)]
				\item $\tau$ is continuous at $f(a)\in J$;
				\item $g(f(x))-g(f(a))=\tau(f(x))(f(x)-f(a))$;
				\item $g'(f(a))=\tau(f(a))$.
			\end{enumerate}
		\end{enumerate} Then \begin{align*}
			g(f(x))-g(f(a))&=\tau(f(x))(f(x)-f(a))\quad\text{by (2)-(ii)}\\
			&=\tau(f(x))\sigma(x)(x-a)\quad\text{by (1)-(ii)}.
		\end{align*} Let $\varphi(x):=\tau(f(x))\sigma(x)$. Then \begin{enumerate}[(i)]
			\item $\phi:I\to\R$ is continuous at $a$ and
			\item $g(f(x))-g(f(a))=\varphi(x)(x-a)$,
		\end{enumerate} and so, by \Caratheodroy's Theorem, \[
		\exists(g\circ f)'(a)=\varphi(a)=\tau(f(a))\cdot \sigma(a)=g'(f(a))\cdot f'(a).
		\]
	\end{proof}
	\vspace{4pt}
	\begin{remark}
		If $f$ is a differentiable function, then the chain rule implies that the function $g\circ f=\abs{f}$ is also differentiable at all points $x$ where $f(x)\neq 0$, and its derivative is given by \[
		\abs{f(x)}'(x)=\mathsf{sgn}(f(x))\cdot f'(x)=\begin{cases}
			f'(x) &:f(x)>0,\\
			-f'(x) &:f(x)<0.
		\end{cases}
		\]
	\end{remark}
	\vspace{4pt}
	\begin{remark}\it
		A function $f$ that is differentiable at every point of $\R$ \textbf{need not} have a continuous derivative $f'$.
	\end{remark}
	
	\newpage
	\begin{tcolorbox}[colback=white,colframe=thmcolor,arc=5pt,title={\color{white}\bf Differentiablility of The Inverse Function}]
		\begin{theorem}
			Let $f:I\to\R$ be strictly monotone and continuous on $I$. Let $J:=f[I]$ and $g:J\to\R$ be the strictly monotone and continuous function inverse to $f$. Then \[
			\exists f'(a)\neq 0\implies\exists g'(f(a))=\frac{1}{f'(a)}.
			\]
		\end{theorem}
	\end{tcolorbox}
	\begin{proof}
		Since $\exists f'(a)$, by \Caratheodroy's Theorem, $\exists\sigma:I\to\R$ s.t. \begin{enumerate}[(i)]
			\item $\sigma$ is continuous at $a\in I$;
			\item $f(x)-f(a)=\sigma(x)(x-a)$;
			\item $f'(a)=\sigma(a)\neq 0$.
		\end{enumerate} Since $\sigma(a)\neq 0$, $\exists\delta>0$ s.t. $\sigma(x)\neq0$, $x\in\nbhd_\delta(a)\cap I$.
		Let $\Omega:=f[\nbhd_\delta(a)\cap I]$. Since $g=f^{-1}$, we have \begin{align*}
			f(\textcolor{red}{x})-f(\textcolor{blue}{a})&=f(\textcolor{red}{(g\circ f)(x)})-f(\textcolor{blue}{(g\circ f)(a)})\quad\because f\circ g=\Id\\
			&=\sigma(\textcolor{red}{(g\circ f)(x)})(\textcolor{red}{(g\circ f)(x)}-\textcolor{blue}{(g\circ f)(a)})\quad\text{by (ii)}.
		\end{align*} Since $f(x)\in\Omega\Rightarrow\sigma(x)\neq 0\Rightarrow \sigma((g\circ f)(x))\neq 0$, \[
		g(f(x))-g(f(a))=\frac{1}{\sigma((g\circ f)(x))}(f(x)-f(a)).
		\] Let $\varphi(x):=1/\sigma((g\circ f)(x))$. Then $\varphi$ is continuous at $f(a)$. By \Caratheodroy's Theorem, \[
		g'(f(a))=\varphi(a)=\frac{1}{\sigma((g\circ f)(a))}=\frac{1}{\sigma(a)}=\frac{1}{f'(a)}.
		\]
	\end{proof}
	
	\newpage
	\section{Mean Value Theorem}
	\begin{tcolorbox}[colback=white,colframe=defcolor,arc=5pt,title={\color{white}\bf }]
		\begin{definition}
			Let $f:I\to\R$ be a function.
			\begin{itemize}
				\item $f$ has an \textbf{absolute maximum} at $a\in I$ if $x\in I\implies f(x)\leq f(a)$.
				\item $f$ has an \textbf{absolute minimum} at $a\in I$ if $x\in I\implies f(a)\leq f(x)$.
				\item $f$ is said to have a \textbf{local (or relative) maximum} at $a\in I$ if \[
				\exists\nbhd_\delta(a):f(x)\leq f(a),\ x\in\nbhd_\delta(a)\cap I.
				\]
				\item $f$ is said to have a \textbf{local (or relative) minimum} at $a\in I$ if \[
				\exists\nbhd_\delta(a):f(a)\leq f(x),\ x\in\nbhd_\delta(a)\cap I.
				\]
				\item $f$ has a \textbf{local (or relative extremum)} at $a\in I$ either a relative maximum or a relative minimum at $a$.
			\end{itemize}
		\end{definition}
	\end{tcolorbox}
	\vspace{8pt}
	\begin{tcolorbox}[colback=white,colframe=thmcolor,arc=5pt,title={\color{white}\bf Interior Extremum Theorem}]
		\begin{theorem}
		Let $f:(a,b)\to\R$ has a relative extremum and $c\in(a,b)$. Then \[
		\exists f'(c)\implies f'(c)=0.
		\]
		\end{theorem}
	\end{tcolorbox}
	\begin{proof}
		Let $f$ has a relative maximum at $c$, \ie, \[
		\exists\nbhd_\delta(a): x\in\nbhd_\delta(a)\cap (a,b)\implies f(x)\leq f(a).
		\] Assume that $f'(c)>0$ then \[
		\exists\nbhd_\delta(c)\subseteq(a,b):x\in\nbhd_\delta^*(c)\Rightarrow\frac{f(x)-f(c)}{x-c}>0.
		\] If $c\in\nbhd_\delta(c)$ and $x>c$, then we have \[
		f(x)-f(c)=(x-c)\cdot\frac{f(x)-f(c)}{x-c}>0.
		\] But this contradicts the hypothesis that $f$ has a relative maximum at $c$. Similarly if $f'(c)<0$ then we have a contradiction. Hence $f'(c)=0$.
	\end{proof}
	\vspace{4pt}
	\begin{tcolorbox}[colback=white,colframe=corcolor,arc=5pt,title={\color{white}\bf }]
		\begin{corollary}
			Let $f:(a,b)\to\R$ be continuous on $(a, b)$ and suppose that $f$ has a relative
			extremum at $c\in(a, b)$. Then either \[
			\nexists f'(c)\quad\text{or}\quad f'(c)=0.
			\]
		\end{corollary}
	\end{tcolorbox}
	\vspace{8pt}
	
	\begin{tcolorbox}[colback=white,colframe=thmcolor,arc=5pt,title={\color{white}\bf $\star$ Rolle's Theorem}]
		\begin{theorem}
			Let $f$ is continuous on $I = [a, b]$, and let $f$ is differentiable on $(a,b)$. Then \[
			f(a)=0=f(b)\implies\exists c\in(a,b):f'(c)=0.
			\]
		\end{theorem}
	\end{tcolorbox}
	\vspace{8pt}
	\begin{tcolorbox}[colback=white,colframe=thmcolor,arc=5pt,title={\color{white}\bf $\star$ Mean Value Theorem of Differential Calculus $\star$}]
		\begin{theorem}
			Let $f$ is continuous on $I = [a, b]$, and let $f$ is differentiable on $(a,b)$. Then \[
			\exists c\in(a,b):f(b)-f(a)=f'(c)(b-a).
			\]
		\end{theorem}
	\end{tcolorbox}
	\begin{proof}
		Consider the function whose graph is the line segment joining the points \((a,f(a))\) and \((b,f(b))\): \[
		f(x)-f(a)=\frac{f(b)-f(a)}{b-a}(x-a).
		\] Define a function \(g:[a,b]\to\R\) s.t. \[
		g(x):=f(x)-f(a)-\frac{f(b)-f(a)}{b-a}(x-a)
		\] Then \begin{enumerate}[(i)]
			\item \(g\) is continuous on \([a,b]\);
			\item \(g\) is differentiable on \((a,b)\);
			\item \(g(a)=0=g(b)\). 
		\end{enumerate} By Rolle's Theorem, \(\exists c\in(a,b):g'(c)=0\). Then \[
		g'(x)=f'(x)-\frac{f(b)-f(a)}{b-a}\implies g'(c)=f'(c)-\frac{f(b)-f(a)}{b-a}=0\implies
		f'(c)=\frac{f(b)-f(a)}{b-a}.
		\]
	\end{proof}
	\vspace{8pt}
	\begin{example}
		Prove that $e^x\geq 1+x$ for $x\in\R$.
		\begin{proof}[\sol]
			\begin{enumerate}[(1)]
				\item \(x=0\implies e^x=1+x\).
				\item Let \(x>0\) and \(f(x)=e^x\). Then, by MVT, \[
				\exists c\in(0,x):f(x)-f(0)=f'(c)(x-0),
				\] and so \[
				e^x-1=e^cx> x\implies e^x>1+x.
				\]
				\item Let $x<0$ and $f(x)=e^x$. Then, by MVT, \[
				\exists c\in(x,0): f(0)-f(x)=f'(c)(0-x),
				\] and so \[
				1-e^x=e^c(-x)< -x\implies 1+x<e^x.
				\]
			\end{enumerate}
		\end{proof}
	\end{example}
	
	\newpage
	
	
	
	
	
	
	
	
	
	
	
	
	
	
	
	
	% End document
\end{document}
